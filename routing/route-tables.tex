\usetikzlibrary{matrix}
\begin{frame}{routing tables}
\begin{tikzpicture}
\matrix[tight matrix,
    nodes={minimum height=.6cm},gT
    column 1/.style={nodes={text width=4.5cm,font=\small\tt}},
    column 2/.style={nodes={text width=2cm,font=\small\tt,alt=<4>{fill=red!10}}},
    column 3/.style={nodes={text width=1cm,font=\small\tt,alt=<3>{fill=red!10}}},
    row 1/.style={nodes={font=\small}},
    label={north:routing table},
    anchor=north west
] (route table v6) {
IP addresses \& gateway \& iface \\
2001:0db8:40:f000:;/44 \& --- \& int1 \\
2001:0db8:40:e000::/44 \& 2001:0db8:40:f000::2 \& int1 \\
2001:0db8:40:d000::/44 \& --- \& int3 \\
3fff:1000:19::/48 \& --- \& ext1 \\
\ldots \& \ldots \& \ldots \\
\normalfont default \& fe80::17 \& ext2 \\
};
\matrix[tight matrix,
    nodes={minimum height=.6cm},gT
    column 1/.style={nodes={text width=4.5cm,font=\small\tt}},
    column 2/.style={nodes={text width=2cm,font=\small\tt,alt=<4>{fill=red!10}}},
    column 3/.style={nodes={text width=1cm,font=\small\tt,alt=<3>{fill=red!10}}},
    row 1/.style={nodes={font=\small}},
    label={north:routing table},
    anchor=north west
] (route table v4) at (route table v6.south west) {
IP addresses \& gateway \& iface \\
192.0.2.0/25 \& --- \& int1 \\
192.0.2.128/26 \& 192.0.2.1 \& int1 \\
192.0.2.192/26 \& 192.0.2.2 \& int1 \\
198.51.100.0/25 \& 192.0.2.1 \& int1 \\
198.51.100.128/25 \& --- \& int2 \\
\ldots \& \ldots \& \ldots \\
\normalfont default \& 203.0.113.1 \& ext \\
};
\end{tikzpicture}
\end{frame}

\begin{frame}{filling routing tables}
    \begin{itemize}
    \item easy part: what networks are you directly connected to
        \begin{itemize}
        \item that range of IP addresses, that interface
        \end{itemize}
    \vspace{.5cm}
    \item harder part: other routers on connected router
    \item need to learn:
        \begin{itemize}
        \item addresses of other router
        \item which networks can be reached through them directly or indirectly
        \end{itemize}
    \item need to choose between multiple ways of reaching networks
    \end{itemize}
\end{frame}
