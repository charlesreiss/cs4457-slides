\usetikzlibrary{arrows.meta,matrix}

\providecommand{\noroute}{$\infty$/---}
\providecommand{\mymark}[2]{\alt<#1->{\myemph<#1>{#2}}{}}

\begin{frame}{split horizon with poison reverse}
    \begin{itemize}
    \item when sending distance vectors, `posion' routes to same node
        \begin{itemize}
        \item make sure other node won't go back to us\ldots
        \item only to have us go back to them
        \end{itemize}
    \item example, if I'm A and routes are:
        \begin{itemize}
        \item A: 0 via A; B: 2 via B; C: 4 via C; D: 6 via B
        \end{itemize}
    \item when sending to B send:
        \begin{itemize}
        \item A: 0; B: 2; C: 4; D: $\infty$
        \end{itemize}
    \end{itemize}
\end{frame}

\begin{frame}{without split horizon?}
    \begin{itemize}
    \item can create routing loop
    \item example: if D unreachable from B, then B goes to A and A goes to B
    \vspace{.5cm}
    \item called ``count-to-infinity'' problem
        \begin{itemize}
        \item because A, B will keep updating distance higher and higher
        \end{itemize}
    \end{itemize}
\end{frame}

\begin{frame}[fragile]{avoided count-to-infinity}
\begin{tikzpicture}
\tikzset{
    computer/.style={inner sep=0mm,outer sep=0mm,execute at begin node={\computer}},
    switch/.style={inner sep=0mm,outer sep=0mm,execute at begin node={\switch}},
    port/.style={pos=0.95,fill=white,circle,draw,inner sep=0mm},
    port beginning/.style={pos=0.05,fill=white,circle,draw,inner sep=0mm},
    route table/.style={
        matrix of nodes,ampersand replacement=\&,
        column 1/.style={nodes={draw,thick,text width=2.5cm,font=\tiny\tt,text depth=0mm,minimum height=0.5cm,inner sep=1mm}},
        column 2/.style={nodes={draw,thick,text width=.5cm,font=\small\tt,text depth=0mm,minimum height=0.5cm,inner sep=1mm}},
        row 1/.style={nodes={draw=none,font=\small}},
    },
    mac label/.style={
        draw,fill=white,inner sep=1mm,font=\tiny\tt,
    },
    n real/.style={solid},
    n real at/.style={alt=<#1->{solid},alt=<#1>{fill=red!10}},
    n real until/.style={alt=<1-#1>{solid}},
    tn/.style={draw,dotted,circle,very thick},
    real at/.style={alt=<#1->{solid},alt=<#1>{fill=red!10}},
    connect/.style={draw,very thick,-Latex},
    connect big/.style={draw,ultra thick,-Latex},
    actual at/.style={alt=<#1->{draw=red,solid},alt=<#1>{draw,solid}},
    consider at/.style={alt=<#1>{draw=red}},
    msg link/.style={draw,violet,line width=1.2mm,-Latex},
    msg data/.style={draw=violet,line width=1mm,fill=white,align=left},
}
\begin{scope}[x=0.8cm]
\node[tn,n real] (A) at (0, 0) {A};
\node[tn,n real] (B) at (5, -2) {B};
\node[tn,n real] (C) at (-5, -2) {C};
\node[tn,n real] (D) at (0, -4) {D};
\draw[connect big] ([yshift=1mm]A.east) -- ([xshift=1mm]B.north) node[midway,above] {2};
\draw[connect] ([xshift=-1mm]B.north) -- ([yshift=-1mm]A.east) node[midway,below] {4};
\draw[connect big] ([yshift=1mm]A.west) -- ([xshift=-1mm]C.north) node[midway,above] {1};
\draw[connect big] ([xshift=1mm]C.north) -- ([yshift=-1mm]A.west) node[midway,below] {2};
\draw[connect big] ([yshift=1mm]C.east) -- ([yshift=1mm]B.west) node[midway,above] {2};
\draw[connect big] ([yshift=-1mm]B.west) -- ([yshift=-1mm]C.east) node[midway,below] {2};
\draw[connect,alt=<2->{dotted,black!20}] ([xshift=1mm]C.south) -- ([yshift=1mm]D.west) node[midway,above] {4};
\draw[connect,alt=<2->{dotted,black!20}] ([yshift=-1mm]D.west) -- ([xshift=-1mm]C.south) node[midway,below] {5};
\draw[connect big,alt=<1>{red},alt=<2->{dotted,black!20}] ([xshift=-1mm]B.south) -- ([yshift=1mm]D.east) node[midway,above] {1};
\draw[connect big,alt=<1>{red},alt=<2->{dotted,black!20}] ([yshift=-1mm]D.east) -- ([xshift=1mm]B.south) node[midway,below] {2};
\tikzset{
    one dv/.style={
        tight matrix,
        nodes={fill=white,font=\fontsize{10}{11}\selectfont,text width=1cm,minimum height=0.5cm}
    }
}
\matrix[anchor=south,one dv,
    row 2/.style={nodes={fill=violet!10}},
] (A dv) at (A.north) {
    A \& B \& C \& D \\
    0/A \& 2/B \& 4/C \& \only<1-2>{3/B}\only<3->{\myemph<3>{9/B}} \\
};
\matrix[anchor=north,one dv,
    row 2/.style={nodes={fill=yellow!10}},
] (D dv) at ([yshift=0cm]D.south) {
    A \& B \& C \& D \\
    \noroute \& \noroute \& \noroute \& 0/D \\
};
\matrix[anchor=south west,one dv,
    row 2/.style={nodes={fill=blue!10}},
] (C dv) at ([xshift=-3cm]C.north east) {
    A \& B \& C \& D \\
    2/A \& 2/B \& 0/C \& \noroute \\
};
\matrix[anchor=north east,one dv,
    row 2/.style={nodes={fill=green!10}},
] (B dv) at ([yshift=.25cm,xshift=4cm]B.south west) {
    A \& B \& C \& D \\
    4/A \& 0/B \& 2/C \& \only<1>{\noroute}\only<2-3>{\myemph<2>{7/A}}\only<4>{\myemph<4>{13/A}} \\
};
\begin{visibleenv}<2>
\draw[msg link] (A) -- (B)
    node[midway,above left,msg data] { I'm A: A=0,B=2,C=4,D=3 };
\end{visibleenv}
\begin{visibleenv}<3>
\draw[msg link] (B) -- (A)
    node[midway,above left,msg data] { I'm B: A=4,B=0,C=2,D=7 };
\end{visibleenv}
\begin{visibleenv}<4>
\draw[msg link] (A) -- (B)
    node[midway,above left,msg data] { I'm A: A=4,B=0,C=2,D=7 };
\end{visibleenv}
\end{scope}
\end{tikzpicture}
\end{frame}

\begin{frame}{trivial loop}
    \begin{itemize}
    \item oops: A to B to A to B to A to B to \ldots
    \vspace{.5cm}
    \item this case: relatively easy to avoid
    \end{itemize}
\end{frame}


\begin{frame}{split horizon incomplete solution}
    \begin{itemize}
    \item split horizon prevents trivial loops, but\ldots
    \item doesn't actually solve the count-to-infinty problem
    \vspace{.5cm}
    \item in well-connected network, there will be longer loops
    \end{itemize}
\end{frame}

\begin{frame}[fragile]{count-to-infinity (v2)}
\begin{tikzpicture}
\tikzset{
    computer/.style={inner sep=0mm,outer sep=0mm,execute at begin node={\computer}},
    switch/.style={inner sep=0mm,outer sep=0mm,execute at begin node={\switch}},
    port/.style={pos=0.95,fill=white,circle,draw,inner sep=0mm},
    port beginning/.style={pos=0.05,fill=white,circle,draw,inner sep=0mm},
    route table/.style={
        matrix of nodes,ampersand replacement=\&,
        column 1/.style={nodes={draw,thick,text width=2.5cm,font=\tiny\tt,text depth=0mm,minimum height=0.5cm,inner sep=1mm}},
        column 2/.style={nodes={draw,thick,text width=.5cm,font=\small\tt,text depth=0mm,minimum height=0.5cm,inner sep=1mm}},
        row 1/.style={nodes={draw=none,font=\small}},
    },
    mac label/.style={
        draw,fill=white,inner sep=1mm,font=\tiny\tt,
    },
    n real/.style={solid},
    n real at/.style={alt=<#1->{solid},alt=<#1>{fill=red!10}},
    n real until/.style={alt=<1-#1>{solid}},
    tn/.style={draw,dotted,circle,very thick},
    real at/.style={alt=<#1->{solid},alt=<#1>{fill=red!10}},
    connect/.style={draw,very thick,-Latex},
    connect big/.style={draw,ultra thick,-Latex},
    actual at/.style={alt=<#1->{draw=red,solid},alt=<#1>{draw,solid}},
    consider at/.style={alt=<#1>{draw=red}},
    msg link/.style={draw,violet,line width=1.2mm,-Latex},
    msg data/.style={draw=violet,line width=1mm,fill=white,align=left},
}
\begin{scope}[x=0.8cm]
\node[tn,n real] (A) at (0, 0) {A};
\node[tn,n real] (B) at (5, -2) {B};
\node[tn,n real] (C) at (-5, -2) {C};
\node[tn,n real] (D) at (0, -4) {D};
\draw[connect big] ([yshift=1mm]A.east) -- ([xshift=1mm]B.north) node[midway,above] {2};
\draw[connect] ([xshift=-1mm]B.north) -- ([yshift=-1mm]A.east) node[midway,below] {4};
\draw[connect big] ([yshift=1mm]A.west) -- ([xshift=-1mm]C.north) node[midway,above] {1};
\draw[connect big] ([xshift=1mm]C.north) -- ([yshift=-1mm]A.west) node[midway,below] {2};
\draw[connect big] ([yshift=1mm]C.east) -- ([yshift=1mm]B.west) node[midway,above] {2};
\draw[connect big] ([yshift=-1mm]B.west) -- ([yshift=-1mm]C.east) node[midway,below] {2};
\draw[connect,alt=<2->{dotted,black!20}] ([xshift=1mm]C.south) -- ([yshift=1mm]D.west) node[midway,above] {4};
\draw[connect,alt=<2->{dotted,black!20}] ([yshift=-1mm]D.west) -- ([xshift=-1mm]C.south) node[midway,below] {5};
\draw[connect big,alt=<1>{red},alt=<2->{dotted,black!20}] ([xshift=-1mm]B.south) -- ([yshift=1mm]D.east) node[midway,above] {1};
\draw[connect big,alt=<1>{red},alt=<2->{dotted,black!20}] ([yshift=-1mm]D.east) -- ([xshift=1mm]B.south) node[midway,below] {2};
\tikzset{
    one dv/.style={
        tight matrix,
        nodes={fill=white,font=\fontsize{10}{11}\selectfont,text width=1cm,minimum height=0.5cm}
    }
}
\matrix[anchor=south,one dv,
    row 2/.style={nodes={fill=violet!10}},
] (A dv) at (A.north) {
    A \& B \& C \& D \\
    0/A \& 2/B \& 4/C \& \only<1-3>{3/B}\only<4->{\myemph<4>{8/B}} \\
};
\matrix[anchor=north,one dv,
    row 2/.style={nodes={fill=yellow!10}},
] (D dv) at ([yshift=0cm]D.south) {
    A \& B \& C \& D \\
    \noroute \& \noroute \& \noroute \& 0/D \\
};
\matrix[anchor=south west,one dv,
    row 2/.style={nodes={fill=blue!10}},
] (C dv) at ([xshift=-3cm]C.north east) {
    A \& B \& C \& D \\
    2/A \& 2/B \& 0/C \& \only<1>{\noroute}\only<2->{\myemph<2>{4/A}} \\
};
\matrix[anchor=north east,one dv,
    row 2/.style={nodes={fill=green!10}},
] (B dv) at ([yshift=.25cm,xshift=4cm]B.south west) {
    A \& B \& C \& D \\
    4/A \& 0/B \& 2/C \& \only<1-2>{\noroute}\only<3->{\myemph<3>{6/C}} \\
};
\begin{visibleenv}<2>
\draw[msg link] (A) -- (C);
\end{visibleenv}
\begin{visibleenv}<3>
\draw[msg link] (C) -- (B);
\end{visibleenv}
\begin{visibleenv}<4>
\draw[msg link] (B) -- (A);
\end{visibleenv}
\end{scope}
\end{tikzpicture}
\end{frame}

\begin{frame}{count-to-infinity}
    \begin{itemize}
    \item when node becomes unreachable, can have `phantom' routes
    \item keep propogating in loop, incrementing metric forever
    \vspace{.5cm}
    \item RIP solution: maximum metric is 15 (hops)
    \end{itemize}
\end{frame}

\begin{frame}{better count-to-infinity solutions?}
    \begin{itemize}
    \item can share information about more than just neighbors
    \vspace{.5cm}
    \item we'll see two examples:
    \item link-state routing protocols (example: OSPF)
        \begin{itemize}
        \item every router learns full map of network
        \end{itemize}
    \item border gateway protocol (BGP)
        \begin{itemize}
        \item (basically) track \textit{list of hops} alongside distances
        \item eliminate potential routes that would create duplicate hops (loops)
        \end{itemize}
    \end{itemize}
\end{frame}
