\begin{frame}{Routing Information Protocol}
\begin{itemize}
\item router broadcast on networks it's connected to packet containing list of:
    \begin{itemize}
    \item networks it can reach (example: 1.2.3.0/24)
    \item its next hop to that network
    \item its metric (distance) to reach that network
    \end{itemize}
\item each router on that network processes that packet
\item on receiving distances, routers see if they can update their routes
    \begin{itemize}
    \item routes will be to networks (1.2.3.0/24, etc.), not routers
    \end{itemize}
\end{itemize}
\end{frame}

\begin{frame}{local information}
    \begin{itemize}
    \item routers need to track themselves:
    \vspace{.5cm}
    \item which networks they can reach directly
        \begin{itemize}
        \item (which networks is it connected to)
        \end{itemize}
    \item the `distance' it needs to reach those networks
        \begin{itemize}
        \item (probably based on its bandwidth to that network?)
        \end{itemize}
    \end{itemize}
\end{frame}

\begin{frame}{RIP --- when to update}
    \begin{itemize}
    \item policy: every approx. 30 seconds always AND
    \item immediately on changes (``triggered'')
    \vspace{.5cm}
    \item means that connecting new router should better routes quickly
    \end{itemize}
\end{frame}
