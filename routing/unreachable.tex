\begin{frame}[fragile]{destination host unreachable}
\begin{Verbatim}
$ ping 128.143.67.254
PING 128.143.67.254 (128.143.67.254) 56(84) bytes of data.
From 128.143.63.1 icmp_seq=1 Destination Host Unreachable
From 128.143.63.1 icmp_seq=6 Destination Host Unreachable
^C
--- 128.143.67.254 ping statistics ---
10 packets transmitted, 0 received, +2 errors, 100% packet loss, time 9146ms
pipe 4
\end{Verbatim}
---
\begin{Verbatim}
$ ping6 2606:8e80:7007:ef1a::1
PING 2606:8e80:7007:ef1a::1(2606:8e80:7007:ef1a::1) 56 data bytes
From 2606:8e80:7007:ef1a:cf1f:3948:b5c1:a522 icmp_seq=1 Destination unreachable: Address unreachable
....
\end{Verbatim}
\end{frame}

\begin{frame}{ICMPv6 destination unreachable messages}
\begin{itemize}
\item IPv6 header with ICMP as next protocol
\item 1 byte type = 1 (destination unreachable)
\item 1 byte code =
    \begin{itemize}
    \item examples: address unreachable, administritatively prohibited
    \end{itemize}
\item most of contents of message causing problem
    \begin{itemize}
    \item only most to avoid exceeding max packet size
    \item should let OS figure out which socket to send error to
    \end{itemize}
\end{itemize}
\end{frame}

\begin{frame}{generating destination unreachable}
    \begin{itemize}
    \item by routers: reached correct network, machine not there
    \item by routers: no route to network at all
    \item by routers: administrator rule prohibits forwarding
    \item by destination host: no program listening to that `port'
    \item \ldots
    \vspace{.5cm}
    \item different code values for all cases
    \item machine can also choose to send nothing back
    \end{itemize}
\end{frame}

\begin{frame}{ICMPv4 destination unrachable}
\begin{itemize}
\item basically same format as ICMPv6, but\ldots
\item different type/code integer values
\item only IPv4 header + 64 bytes of original packet included
\end{itemize}
\end{frame}
