
\begin{frame}<1>[label=controlToC]{P4: control plane}
    \begin{itemize}
    \item P4 control plane is a program
    \vspace{.5cm}
    \item sends commands to one or more switches:
        \begin{itemize}
        \item \myemph<2>{load P4 program into data plane}
        \item \myemph<3>{set table entries}
        \item \myemph<4>{configure multicast groups}
        \item \myemph<5>{receive frames to process them}
        \end{itemize}
    \end{itemize}
\end{frame}

\begin{frame}{aside: wrappers}
    \begin{itemize}
    \item I'll show code from a P4 controller for upcoming assignment
    \item written in Python, using custom library to make things convenient
    \item works with softawre based reference switch
    \vspace{.5cm}
    \item no requirement to use Python or other specific language
        \begin{itemize}
        \item controller sends commands over network/IPC to data plane
        \end{itemize}
    \item real raw code has more boilerplate/etc.
    \item probably several things different for hardware-based switches
    \end{itemize}
\end{frame}

\againframe<2>{controlToC}

\begin{frame}[fragile]{P4: loading P4 program}
\providecommand{\vemphA}[1]{\myemph<2>{#1}}
\providecommand{\vemphB}[1]{\myemph<3>{#1}}
\providecommand{\vemphC}[1]{\myemph<4>{#1}}
\providecommand{\vemphD}[1]{\myemph<5>{#1}}
\begin{Verbatim}[fontsize=\small,commandchars=\\@~]
p4info_helper =  ....
switch = ....
\vemphA@switch.MasterArbitrationUpdate()~
switch.\vemphB@SetForwardingPipelineConfig~(
    p4info=p4info_helper.p4info,
    \vemphC@bmv2_json_file_path=bmv2_file_path~
)
\end{Verbatim}
\begin{tikzpicture}[overlay,remember picture]
\tikzset{
    box/.style={
        align=left,
        draw=red,
        fill=white,
        ultra thick,
        anchor=north,
        at={([yshift=-1cm]current page.center)},
    },
}
    \begin{visibleenv}<2>
    \node[box] {
        swtich supports having primary + backup controller, \\
        so need to indicate this is primary controller now \\
    };
    \end{visibleenv}
    \begin{visibleenv}<3>
    \node[box] {
        ``forwarding pipeline'' = dataplane
    };
    \end{visibleenv}
    \begin{visibleenv}<4>
    \node[box] {
        P4 code compiled to file to load, specified here
    };
    \end{visibleenv}
\end{tikzpicture}
\end{frame}

\againframe<3>{controlToC}

\begin{frame}[fragile]{P4: setting table entries}
\providecommand{\vemphA}[1]{\myemph<2>{#1}}
\providecommand{\vemphB}[1]{\myemph<3>{#1}}
\providecommand{\vemphC}[1]{\myemph<4>{#1}}
\providecommand{\vemphD}[1]{\myemph<5>{#1}}
\begin{Verbatim}[fontsize=\small,commandchars=\\@~]
\vemphD@write_or_overwrite_table_entry~(
    \vemphA@p4info_helper=p4info_helper, switch=switch,~
    table_name='\vemphB@MyIngress.mac_dst_exact~',
    match_fields={
        'hdr.ethernet.dstAddr': \vemphC@some_address~,
    },
    action_name='forward_to',
    action_params={'port': port},
)
\end{Verbatim}
\begin{tikzpicture}[overlay,remember picture]
\tikzset{
    box/.style={
        align=left,
        draw=red,
        fill=white,
        ultra thick,
        anchor=north,
        at={([yshift=-1cm]current page.center)},
    },
}
    \begin{visibleenv}<2>
    \node[box] {
        p4info\_helper, switch objects created by setup code
    };
    \end{visibleenv}
    \begin{visibleenv}<3>
    \node[box] {
        full name of table, including stage it is defined in
    };
    \end{visibleenv}
    \begin{visibleenv}<4>
    \node[box] {
        match value --- format would be different \\
        if key was \texttt{lpm} or \texttt{ternary} \\
        instead of exact match
    };
    \end{visibleenv}
    \begin{visibleenv}<5>
    \node[box] {
        \texttt{write\_or\_overwrite\_table\_entry} not the `raw' function \\
        (one I wrote based on one P4 tutorial authors wrote) \\
        uses gRPC (remote procedure call) library, which adds some extra steps
    };
    \end{visibleenv}
\end{tikzpicture}
\end{frame}

\againframe<4>{controlToC}
\begin{frame}[fragile]{P4: multicast groups}
\providecommand{\vemphA}[1]{\myemph<2>{#1}}
\providecommand{\vemphB}[1]{\myemph<3>{#1}}
\providecommand{\vemphC}[1]{\myemph<4>{#1}}
\providecommand{\vemphD}[1]{\myemph<5>{#1}}
\begin{Verbatim}[fontsize=\small,commandchars=\\@~]
switch.Write\vemphC@PRE~Entry(
    p4info_helper.buildMulticastGroupEntry(
        \vemphA@multicast_group_id=1~, 
        \vemphB@replicas~=[
            {'egress_port': 1, 'instance': 0},
            {'egress_port': 2, 'instance': 0},
            {'egress_port': 3, 'instance': 0},
            {'egress_port': 4, 'instance': 0},
        ]
    )
)
\end{Verbatim}
\begin{tikzpicture}[overlay,remember picture]
\tikzset{
    box/.style={
        align=left,
        draw=red,
        fill=white,
        ultra thick,
        anchor=north,
        at={([yshift=-1cm]current page.center)},
    },
}
    \begin{visibleenv}<2>
    \node[box] {
        in P4 dataplane code can write \\
        \texttt{standard\_metadata.mcast\_grp = 1} to use this
    };
    \end{visibleenv}
    \begin{visibleenv}<3>
    \node[box] {
        list of ports to output to when group selected \\
        ~ \\
        \texttt{instance} can be inspected by dataplane code \\
        for the egress step
    };
    \end{visibleenv}
    \begin{visibleenv}<4>
    \node[box] {
        PRE = packet replication engine \\
        ~ \\
        supports multicast groups (shown) and ``clone sessions'' (not shown) \\
        (clone sessions make extra copy of packet, \\
        but process original normally)
    };
    \end{visibleenv}
\end{tikzpicture}
\end{frame}

\againframe<5>{controlToC}
\begin{frame}[fragile]{P4: receiving frame}
\providecommand{\vemphA}[1]{\myemph<2>{#1}}
\providecommand{\vemphB}[1]{\myemph<3>{#1}}
\providecommand{\vemphC}[1]{\myemph<4>{#1}}
\providecommand{\vemphD}[1]{\myemph<5>{#1}}
\begin{Verbatim}[fontsize=\small,commandchars=\\@~]
    for item in switch.stream_msg_resp:
        if item.HasField('packet'):
            do_something_with(\vemphA@item.packet.payload~,
                              \vemphB@item.packet.metadata~)
\end{Verbatim}
\begin{tikzpicture}[overlay,remember picture]
\tikzset{
    box/.style={
        align=left,
        draw=red,
        fill=white,
        ultra thick,
        anchor=north,
        at={([yshift=0cm]current page.center)},
    },
}
    \begin{visibleenv}<2>
    \node[box] {
        payload = bytes of frame  \\
        (what would normally be sent on network)
    };
    \end{visibleenv}
    \begin{visibleenv}<3>
    \node[box] {
        extra metadata can be set by dataplane \\
        example: which port packet came from
    };
    \end{visibleenv}
\end{tikzpicture}
\end{frame}
