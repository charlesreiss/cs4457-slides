\usetikzlibrary{arrows.meta,decorations.pathreplacing}


\begin{frame}[fragile]{P4: declaring headers (1)}
\providecommand{\vemphA}[1]{\myemph<2>{#1}}
\providecommand{\vemphB}[1]{\myemph<3>{#1}}
\providecommand{\vemphC}[1]{\myemph<4>{#1}}
\providecommand{\vemphD}[1]{\myemph<5>{#1}}
\begin{Verbatim}[fontsize=\small,commandchars=\\@~]
typedef bit<48> macAddr_t;
\vemphA@@header~ ethernet_t~ {
    macAddr_t srcAddr;
    macAddr_T dstAddr;
    bit<16>   etherType;
}
\end{Verbatim}
\begin{itemize}
\item<2-> kinda like \Verb|typedef struct { ... } ethernet_t;|
\item<3-> order needs to \textit{exactly} match what's sent on network
    \begin{itemize}
    \item switch will do bit-by-bit copy into this struct
    \end{itemize}
\end{itemize}
\end{frame}

\begin{frame}[fragile]{P4: declaring headers (2)}
\providecommand{\vemphA}[1]{\myemph<2>{#1}}
\providecommand{\vemphB}[1]{\myemph<3>{#1}}
\providecommand{\vemphC}[1]{\myemph<4>{#1}}
\providecommand{\vemphD}[1]{\myemph<5>{#1}}
\begin{Verbatim}[fontsize=\small,commandchars=\\@~]
\vemphA@struct headers~ {
    \vemphB@ethernet_t~   ethernet;
    \vemphB@ipv4_t~       ipv4;
    \vemphB@ipv6_t~       ipv6;
}
\end{Verbatim}
\begin{itemize}
\item<2-> struct of all possible headers
    \begin{itemize}
    \item from all layers the switch/router handles
    \item not all frames will use all of them, some may be mutually exclusive
    \item order does not need to match frame storage (will write code to handle that)
    \end{itemize}
\item<3-> references header types defined previously
\end{itemize}
\end{frame}

% FIXME: hilite parts with explanation
\begin{frame}[fragile]{P4: parsing}
\begin{Verbatim}
parser MyParser(
    packet_in packet,
    out headers hdr,
    inout metadata meta,
    inout standard_metadata_t standard_metadata) {
\end{Verbatim}
\begin{itemize}
\item function modifies `out' parameters instead of returning
\item parser function --- takes parameters:
    \begin{itemize}
    \item packet --- the input frame
    \item headers --- extracted headers (\texttt{struct headers})
    \item meta --- program's local variables about frame (\texttt{struct metadata})
    \item standard\_metadata --- info for system about frame
        \begin{itemize}
        \item where frame came from, where it should go, etc.
        \end{itemize}
    \end{itemize}
\end{itemize}
\end{frame}

\begin{frame}[fragile]{P4: parsing}
\providecommand{\vemphA}[1]{\myemph<2>{#1}}
\providecommand{\vemphB}[1]{\myemph<3>{#1}}
\providecommand{\vemphC}[1]{\myemph<4>{#1}}
\providecommand{\vemphD}[1]{\myemph<5>{#1}}
\begin{Verbatim}[fontsize=\small,commandchars=\\@~]
parser MyParser(...) {
    \vemphA@state~ start {
        \vemphB@transition~ \vemphC@parse_ethernet~;
    }
    \vemphA@state~ \vemphC@parse_ethernet~ {
        \vemphD@packet.extract(hdr.ethernet);~
        \vemphB@transition~ select(hdr.ethernet.etherType) {
            TYPE_IPV4: parse_ipv4;
            TYPE_IPV6: parse_ipv6
            default: accept;
        }
    }
    ....
}
\end{Verbatim}
\begin{tikzpicture}[overlay,remember picture]
    \begin{visibleenv}<2>
    \node[align=left,draw=red,fill=white,very thick] at (current page.center) {
        functions written as state machine \\
        exectuion starts in state \texttt{start} \\
        follows instructions in each state
    };
    \end{visibleenv}
    \begin{visibleenv}<3-4>
    \node[align=left,draw=red,fill=white,very thick] at ([yshift=-1cm]current page.center) {
        transition statements say which state to go to next  \\
        can be conditional using switch-statement-like syntax
    };
    \end{visibleenv}
    \begin{visibleenv}<5>
    \node[align=left,draw=red,fill=white,very thick] at ([yshift=-1.5cm]current page.center) {
        \texttt{extract} operation copies from packet \\
        into specific \texttt{header} object
    };
    \end{visibleenv}
\end{tikzpicture}
\end{frame}
