\begin{frame}{software defined networking (SDN)}
    \begin{itemize}
    \item movement toward \textit{programmable} networks
    \vspace{.5cm}
    \item ``software-defined''
        \begin{itemize}
        \item rules about how network works defined in ``normal'' software
        \end{itemize}
    \end{itemize}
\end{frame}

\begin{frame}{control plane and data plane}
    \begin{itemize}
    \item control plane
        \begin{itemize}
        \item decides \textit{how} to handle traffic
        \item ``slow path'', where complicated decisions are
        \end{itemize}
    \item data plane
        \begin{itemize}
        \item actually implements the decisions made by the control plane
        \item ``fast path'', implementing simple rules
        \end{itemize}
    \vspace{.5cm}
    \item probably what switches did internally before SDN was a thing
    \end{itemize}
\end{frame}

\begin{frame}{separate control/data plane}
    \begin{itemize}
    \item one SDN key idea: separate control and data plane
    \item allow new \myemph<2>{vendor-neutral} implementations of control plane
        \begin{itemize}
        \item requires standard interface for programming data plane
        \item most prominent example: OpenFlow
        \end{itemize}
    \item easily allows for central `control plane' server
        \begin{itemize}
        \item instead of separate control plane running on each switch/router
        \end{itemize}
    \end{itemize}
\end{frame}
