
\usetikzlibrary{arrows.meta,decorations.pathreplacing}
\begin{frame}[fragile]{P4: parsing}
\providecommand{\vemphA}[1]{\myemph<2>{#1}}
\providecommand{\vemphB}[1]{\myemph<3>{#1}}
\providecommand{\vemphC}[1]{\myemph<4>{#1}}
\providecommand{\vemphD}[1]{\myemph<5>{#1}}
\begin{Verbatim}[fontsize=\small,commandchars=\\@~]
parser MyParser(...) {
    ...
    state parse_ethernet {
        \vemphA@packet.extract(hdr.ethernet);~
        transition select(hdr.ethernet.etherType) {
            TYPE_IPV4: parse_ipv4;
            TYPE_IPV6: parse_ipv6
            default: accept;
        }
    }

    state parse_ipv4 {
        \vemphA@packet.extract(hdr.ipv4)~
        transition accept;
    }
    ....
}
\end{Verbatim}
\begin{tikzpicture}[overlay,remember picture]
    \begin{visibleenv}<2>
    \node[align=left,draw=red,fill=white,ultra thick] at ([yshift=-1cm,xshift=4cm]current page.center) {
        extract operations \\
        can be chained \\
        to extract multiple headers \\
        (for multiple layers \\
        or variable-length headers)
    };
    \end{visibleenv}
\end{tikzpicture}
\end{frame}

\begin{frame}[fragile]{P4: deparsing}
\providecommand{\vemphA}[1]{\myemph<2>{#1}}
\providecommand{\vemphB}[1]{\myemph<3>{#1}}
\providecommand{\vemphC}[1]{\myemph<4>{#1}}
\providecommand{\vemphD}[1]{\myemph<5>{#1}}
\begin{Verbatim}[fontsize=\small,commandchars=\\@~]
control MyDeparser(packet_out packet, in headers hdr) {
    apply {
        \vemphA@packet.emit~(hdr.ethernet);
        \vemphA@packet.emit~(hdr.ipv4);
        \vemphA@packet.emit~(hdr.ipv6);
    }
}
\end{Verbatim}
\begin{tikzpicture}[overlay,remember picture]
    \begin{visibleenv}<2>
    \node[align=left,draw=red,fill=white,ultra thick] at ([yshift=-2cm]current page.center) {
        use emit to put extracted bits back in packet \\
        ~  \\
        if header is not marked valid, emit does nothing \\
        (\texttt{extract(hdr.foo)} marks \texttt{hdr.foo} valid, \\
        can also use \texttt{hdr.foo.setValid()} and undo with \\
        \texttt{hdr.foo.setInvalid()})
    };
    \end{visibleenv}
\end{tikzpicture}
\end{frame}
