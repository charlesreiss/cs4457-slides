\usetikzlibrary{arrows.meta,decorations.pathreplacing}


\begin{frame}[fragile]{P4: ingress processing}
\providecommand{\vemphA}[1]{\myemph<2>{#1}}
\providecommand{\vemphB}[1]{\myemph<3>{#1}}
\providecommand{\vemphC}[1]{\myemph<4>{#1}}
\providecommand{\vemphD}[1]{\myemph<5>{#1}}
\begin{itemize}
\item example: send \textit{everything} to output port number 4
\end{itemize}
\begin{Verbatim}[fontsize=\small,commandchars=\\@~]
control MyIngress(...) {
    apply {
        \vemphA@standard_metadata.egress_spec = 4;~
    }
}
\end{Verbatim}
\begin{tikzpicture}[overlay,remember picture]
    \begin{visibleenv}<2>
    \node[align=left,draw=red,fill=white,ultra thick] at ([yshift=-2cm]current page.center) {
        standard\_metadata --- `hard-coded' phase reads \texttt{egress\_spec} \\
        ~ \\
        special \texttt{egress\_spec} value for ``discard frame''
    };
    \end{visibleenv}
\end{tikzpicture}
\end{frame}

\begin{frame}{switch decisions}
    \begin{itemize}
    \item most decisions switches/routers make are table lookups
    \vspace{.5cm}
    \item take field from packet (e.g. destination)
    \item lookup in table what to do with that
        (e.g. send to port X, throw error, etc.)
    \vspace{.5cm}
    \item P4 has special syntax for tables and table lookups
    \end{itemize}
\end{frame}

\begin{frame}[fragile]{P4: tables}
\providecommand{\vemphA}[1]{\myemph<2>{#1}}
\providecommand{\vemphB}[1]{\myemph<3>{#1}}
\providecommand{\vemphC}[1]{\myemph<4>{#1}}
\providecommand{\vemphD}[1]{\myemph<5>{#1}}
\begin{Verbatim}[fontsize=\small,commandchars=\\@~]
\vemphA@table~ mac_dst_exact {
    \vemphB@key~ = {
        \vemphB@hdr.ethernet.dstAddr~ : \vemphC@exact~;
    }
    \vemphB@actions~ = {
        drop,
        forward_to,
    }
    size = 1024;
    default_action = drop();
}
\end{Verbatim}
\begin{tikzpicture}[overlay,remember picture]
\tikzset{
    box/.style={
        align=left,
        draw=red,
        fill=white,
        ultra thick,
        anchor=west,
        at={([xshift=1cm]current page.center)},
    },
}
    \begin{visibleenv}<2>
    \node[box] {
        declare a table with a name \\
        done has part of ingress/outgress
    };
    \end{visibleenv}
    \begin{visibleenv}<3>
    \node[box] {
        key/value structure \\
        keys usually from headers \\
        values are actions to run
    };
    \end{visibleenv}
    \begin{visibleenv}<4>
    \node[box] {
        here, each table entry \\
        contains whole address \\
        P4 also supports \\
        partially matched keys
    };
    \end{visibleenv}
\end{tikzpicture}
\end{frame}

\begin{frame}{P4 table key types}
    \begin{itemize}
    \item exact --- full entry
    \item lpm --- longest prefix match
        \begin{itemize}
        \item table entries contain `prefixes' of header field(s)
        \item we'll see motivation for this when we talk about routing
        \item if multiple entries match, take longest one
        \end{itemize}
    \item ternary --- 0/1/don't care
        \begin{itemize}
        \item table entries contain key and mask
        \item entry matches if bits included in mask match
        \item other view: keys represented with 0/1/don't care `trits'
        \end{itemize}
    \end{itemize}
\end{frame}

\begin{frame}{content-addressable memory}
    \begin{itemize}
    \item high-end routers/switches have \textit{content addressable memory} (CAM)
    \item \ldots including \textit{ternary content addressable memory} (TCAM)
    \vspace{.5cm}
    \item = hardware that implements a table lookup
        \begin{itemize}
        \item `content' $\approx$ key being looked up
        \item looks kinda like highly associative cache
        \item probably lots of comparators
        \end{itemize}
    \item allows multigigabit frame processing speeds
    \item P4 goal: P4 programs compile to use CAM/TCAM when available
    \end{itemize}
\end{frame}

\begin{frame}{aside: TCAM cost}
    \begin{itemize}
    \item 2.7x more transistors than SRAM (common cache technology)
    \item probably much more power than SRAM
    \item hard to find recent quotes for price/power
    \item but probably 10s of dollars per megabyte
        \begin{itemize}
        \item e.g., secondary market price of Broadcom chip with 40Mbit of this is $\sim$\$180
        \item chip needs 80W heatsink, max 900MHz clock rate
        \item but chip has a bunch of other functionality, of course
        \end{itemize}
    \end{itemize}
\end{frame}

\begin{frame}[fragile]{P4: using a table}
\providecommand{\vemphA}[1]{\myemph<2>{#1}}
\providecommand{\vemphB}[1]{\myemph<3>{#1}}
\providecommand{\vemphC}[1]{\myemph<4>{#1}}
\providecommand{\vemphD}[1]{\myemph<5>{#1}}
\begin{Verbatim}[fontsize=\small,commandchars=\\@~]
control MyIngress(...) {
    ...
    table mac_dst_exact = { 
        /* seen earlier */ ...
    }
    apply {
        \vemphA@mac_dst_exact.apply();~
    }
}
\end{Verbatim}
\begin{tikzpicture}[overlay,remember picture]
\tikzset{
    box/.style={
        align=left,
        draw=red,
        fill=white,
        ultra thick,
        anchor=west,
        at={([xshift=1cm]current page.center)},
    },
}
    \begin{visibleenv}<2>
    \node[box] {
        apply operation invokes the action specified by the table
    };
    \end{visibleenv}
\end{tikzpicture}
\end{frame}

\begin{frame}[fragile]{P4: actions}
\providecommand{\vemphA}[1]{\myemph<2>{#1}}
\providecommand{\vemphB}[1]{\myemph<3>{#1}}
\providecommand{\vemphC}[1]{\myemph<4>{#1}}
\providecommand{\vemphD}[1]{\myemph<5>{#1}}
\begin{Verbatim}[fontsize=\small,commandchars=\\@~]
control MyIngress(...) {
    \vemphA@action drop()~ {
        mark_to_drop(\vemphB@standard_metadata~);
    }
    \vemphA@action forward_to(egressSpec_t port)~ {
        \vemphB@standard_metadata.egress_spec = port;~
    }
    \vemphA@action mark_and_forward(egressSpec_t port)~ {
        \vemphC@hdr.some_proto.value~ = 1;
        \vemphB@standard_metadata.egress_spec = port;~
    }
    ...
}
\end{Verbatim}
\begin{tikzpicture}[overlay,remember picture]
\tikzset{
    box/.style={
        align=left,
        draw=red,
        fill=white,
        ultra thick,
        anchor=north,
        at={([yshift=-1cm]current page.center)},
    },
}
    \begin{visibleenv}<2>
    \node[box] {
        actions basically function calls for packet \\
        can include parameters that are stored in table
    };
    \end{visibleenv}
    \begin{visibleenv}<3>
    \node[box] {
        typically, actions set standard metadata \\
        to indicate where to send frame next \\
        ~ \\
        actual sending/not sending frame done later
    };
    \end{visibleenv}
    \begin{visibleenv}<4>
    \node[box] {
        actions can also edit packet headers \\
        or do other table lookups \\
        (which we will need in the future) 
    };
    \end{visibleenv}
\end{tikzpicture}
\end{frame}

\begin{frame}[fragile]{P4: special actions}
    \begin{itemize}
    \item some ways switch can direct packet (incomplete list)
    \vspace{.5cm}
    \item send to the control plane
        \begin{itemize}
        \item special output port that goes to `general purpose' CPU
        \end{itemize}
    \item multicast/broadcast to multiple ports
        \begin{itemize}
        \item control plane can set \textit{multicast groups}
        \item makes multiple copies of packets
        \end{itemize}
    \end{itemize}
\end{frame}

