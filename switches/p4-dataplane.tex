\usetikzlibrary{arrows.meta,decorations.pathreplacing}

\begin{frame}{P4}
    \begin{itemize}
    \item P4 --- programming langauge for data planes
    \item intended to be compiled to run on fast switches
    \item includes `runtime' defining how control plane configures data plane
    \end{itemize}
\end{frame}

\begin{frame}{future P4 assignment}
    \begin{itemize}
    \item given: 
        \begin{itemize}
        \item simple P4 switch that doesn't know where to direct frames
        \item simpler controller (in Python) that configures switch
        \item simulated 4-machine network in VM
        \end{itemize}
    \item your task will be:
        \begin{itemize}
        \item have controller write static configuration to direct to right place
        \item modify data plane to send information to control plane
        \item have controller change configuration based on info from data plane
        \end{itemize}
    \item (we'll discuss more details later)
    \end{itemize}
\end{frame}

\begin{frame}[fragile,label=p4arch]{P4 switch architecture}
\begin{tikzpicture}
\tikzset{
    queue/.style={alt=<2>{fill=red!10,draw=red,ultra thick}},
    explain box/.style={
        align=left,at={(explain box top)},anchor=north,draw=red,very thick
    },
}
\begin{scope}[x=.7cm]
\foreach \x in {1,2,3,4} {
    \node[anchor=east] at (0, \x-.5) {IN$\rightarrow$};
    \node[anchor=west] at (17.5, \x-.5) {$\rightarrow$OUT};
    \foreach \noff in {0,8,16} {
        \begin{scope}[shift={(\noff, 0)},yshift=\x cm-1cm,y=0.9cm]
            \draw[queue,thick] (0, 0) rectangle (1.5, 1);
            \foreach \off in {.5,.7,.9,1.1,1.3} {
                \draw[queue,thick] (\off, 0) -- (\off, 1);
            }
        \end{scope}
    }
}
\coordinate (explain box top) at (8.5, -.7);
\tikzset{
    box/.style={draw,thick},
    box label/.style={font=\fontsize{9}{10}\selectfont},
    parserBox/.style={alt=<6>{fill=red!10}},
    deparserBox/.style={alt=<7>{fill=red!10}},
    ingressBox/.style={alt=<8>{fill=red!10}},
    egressBox/.style={alt=<8>{fill=red!10}},
};
\foreach \x/\name in {2/parser,4/ingress,6/deparser} {
    \draw[box,\name Box] (\x, 0) coordinate (\name-in-base) rectangle ++(1.6, 4) coordinate (\name-in-other-base) node[midway,box label,alias=x] (\name-in-label) {\name};
    \coordinate (\name-in-bottom-right) at (\name-in-base);
    \coordinate (\name-in-bottom-left) at (\name-in-base -| \name-in-other-base);
    \draw[box] (\x + 1.4, -.1) |- (\x - .2, .8-1) |- (\x-.1, 3.8); 
    \draw[box] (\x + 1.5, 0) |- (\x - .1, .9-1) |- (\x, 3.9); 
}
\foreach \x/\name in {10/parser,12/egress,14/deparser} {
    \draw[box,\name Box] (\x, 0) coordinate (\name-out-base) rectangle ++(1.6, 4) coordinate (\name-out-other-base) node[midway,box label] (\name-out label) {\name};
    \coordinate (\name-out-bottom-right) at (\name-out-base);
    \coordinate (\name-out-bottom-left) at (\name-out-base -| \name-out-other-base);
    \draw[box] (\x + 1.4, -.1) |- (\x - .2, .8-1) |- (\x-.1, 3.8); 
    \draw[box] (\x + 1.5, 0) |- (\x - .1, .9-1) |- (\x, 3.9); 
}
\end{scope}
\begin{visibleenv}<2>
    \node[explain box] {
        queues of frames to be processed \\
        if needed, hold packets in between processing steps 
    };
\end{visibleenv}
\begin{visibleenv}<3>
    \foreach \x in {in,out} {
        \draw[line width=.5mm,red,decorate,decoration={brace,mirror,raise=3mm,amplitude=2mm}]
            ([xshift=-.2cm]parser-\x-bottom-right) --
            ([xshift=.2cm]deparser-\x-bottom-left);
    }
    \node[explain box] {
        \textit{ingress} and \textit{egress} pipelines
    };
\end{visibleenv}
\begin{visibleenv}<4>
    \foreach \x in {in} {
        \draw[line width=.5mm,red,decorate,decoration={brace,mirror,raise=3mm,amplitude=2mm}]
            ([xshift=-.2cm]parser-\x-bottom-right) --
            ([xshift=.2cm]deparser-\x-bottom-left);
    }
    \node[explain box] {
        ingress pipeline: processes every input frame \\
        primary job: decide where to forward frame (if anywhere) 
    };
\end{visibleenv}
\begin{visibleenv}<5>
    \foreach \x in {out} {
        \draw[line width=.5mm,red,decorate,decoration={brace,mirror,raise=3mm,amplitude=2mm}]
            ([xshift=-.2cm]parser-\x-bottom-right) --
            ([xshift=.2cm]deparser-\x-bottom-left);
    }
    \node[explain box] {
        egress pipeline: process every output frame \\
        run one time for \myemph{each copy} output
    };
\end{visibleenv}
\begin{visibleenv}<6>
    \foreach \x in {in,out} {
        \draw[line width=.5mm,red,decorate,decoration={brace,mirror,raise=3mm,amplitude=2mm}]
            ([xshift=-.2cm]parser-\x-bottom-right) --
            ([xshift=.2cm]parser-\x-bottom-left);
    }
    \node[explain box] {
        parser: decode frame fields into temporary storage
    };
\end{visibleenv}
\begin{visibleenv}<7>
    \foreach \x in {in,out} {
        \draw[line width=.5mm,red,decorate,decoration={brace,mirror,raise=3mm,amplitude=2mm}]
            ([xshift=-.2cm]deparser-\x-bottom-right) --
            ([xshift=.2cm]deparser-\x-bottom-left);
    }
    \node[explain box] {
        deparser: converted decoded fields back into bytes for network
    };
\end{visibleenv}
\begin{visibleenv}<8>
    \foreach \x in {ingress-in,egress-out} {
        \draw[line width=.5mm,red,decorate,decoration={brace,mirror,raise=3mm,amplitude=2mm}]
            ([xshift=-.2cm]\x-bottom-right) --
            ([xshift=.2cm]\x-bottom-left);
    }
    \node[explain box] {
        ``match/action pipelines'' \\
        do series of table lookups based on parsed fields \\
        each table lookup specifies next action
    };
\end{visibleenv}
\end{tikzpicture}
\end{frame}

\begin{frame}[fragile]{P4: switch parts}
\begin{Verbatim}
V1Switch(
MyParser(),
MyVerifyChecksum(),
MyIngress(),
MyEgress(),
MyComputeChecksum(),
MyDeparser()
) main;
\end{Verbatim}
\begin{itemize}
\item code to declare instance of dataplane from functions for each part
\item will look at most of these components
    \begin{itemize}
    \item won't have reason to change verify/compute checksum
    \end{itemize}
\end{itemize}
\end{frame}

\begin{frame}[fragile]{P4: declaring headers (1)}
\providecommand{\vemphA}[1]{\myemph<2>{#1}}
\providecommand{\vemphB}[1]{\myemph<3>{#1}}
\providecommand{\vemphC}[1]{\myemph<4>{#1}}
\providecommand{\vemphD}[1]{\myemph<5>{#1}}
\begin{Verbatim}[fontsize=\small,commandchars=\\@~]
typedef bit<48> macAddr_t;
\vemphA@@header~ ethernet_t~ {
    macAddr_t srcAddr;
    macAddr_T dstAddr;
    bit<16>   etherType;
}
\end{Verbatim}
\begin{itemize}
\item<2-> kinda like \Verb|typedef struct { ... } ethernet_t;|
\item<3-> order needs to \textit{exactly} match what's sent on network
    \begin{itemize}
    \item switch will do bit-by-bit copy into this struct
    \end{itemize}
\end{itemize}
\end{frame}

\begin{frame}[fragile]{P4: declaring headers (2)}
\providecommand{\vemphA}[1]{\myemph<2>{#1}}
\providecommand{\vemphB}[1]{\myemph<3>{#1}}
\providecommand{\vemphC}[1]{\myemph<4>{#1}}
\providecommand{\vemphD}[1]{\myemph<5>{#1}}
\begin{Verbatim}[fontsize=\small,commandchars=\\@~]
\vemphA@struct headers~ {
    \vemphB@ethernet_t~   ethernet;
    \vemphB@ipv4_t~       ipv4;
    \vemphB@ipv6_t~       ipv6;
}
\end{Verbatim}
\begin{itemize}
\item<2-> struct of all possible headers
    \begin{itemize}
    \item from all layers the switch/router handles
    \item not all frames will use all of them, some may be mutually exclusive
    \item order does not need to match frame storage (will write code to handle that)
    \end{itemize}
\item<3-> references header types defined previously
\end{itemize}
\end{frame}

% FIXME: hilite parts with explanation
\begin{frame}[fragile]{P4: parsing}
\begin{Verbatim}
parser MyParser(
    packet_in packet,
    out headers hdr,
    inout metadata meta,
    inout standard_metadata_t standard_metadata) {
\end{Verbatim}
\begin{itemize}
\item function modifies `out' parameters instead of returning
\item parser function --- takes parameters:
    \begin{itemize}
    \item packet --- the input frame
    \item headers --- extracted headers (\texttt{struct headers})
    \item meta --- program's local variables about frame (\texttt{struct metadata})
    \item standard\_metadata --- info for system about frame
        \begin{itemize}
        \item where frame came from, where it should go, etc.
        \end{itemize}
    \end{itemize}
\end{itemize}
\end{frame}

\begin{frame}[fragile]{P4: parsing}
\providecommand{\vemphA}[1]{\myemph<2>{#1}}
\providecommand{\vemphB}[1]{\myemph<3>{#1}}
\providecommand{\vemphC}[1]{\myemph<4>{#1}}
\providecommand{\vemphD}[1]{\myemph<5>{#1}}
\begin{Verbatim}[fontsize=\small,commandchars=\\@~]
parser MyParser(...) {
    \vemphA@state~ start {
        \vemphB@transition~ \vemphC@parse_ethernet~;
    }
    \vemphA@state~ \vemphC@parse_ethernet~ {
        \vemphD@packet.extract(hdr.ethernet);~
        \vemphB@transition~ select(hdr.ethernet.etherType) {
            TYPE_IPV4: parse_ipv4;
            TYPE_IPV6: parse_ipv6
            default: accept;
        }
    }
    ....
}
\end{Verbatim}
\begin{tikzpicture}[overlay,remember picture]
    \begin{visibleenv}<2>
    \node[align=left,draw=red,fill=white,very thick] at (current page.center) {
        functions written as state machine \\
        exectuion starts in state \texttt{start} \\
        follows instructions in each state
    };
    \end{visibleenv}
    \begin{visibleenv}<3-4>
    \node[align=left,draw=red,fill=white,very thick] at ([yshift=-1cm]current page.center) {
        transition statements say which state to go to next  \\
        can be conditional using switch-statement-like syntax
    };
    \end{visibleenv}
    \begin{visibleenv}<5>
    \node[align=left,draw=red,fill=white,very thick] at ([yshift=-1.5cm]current page.center) {
        \texttt{extract} operation copies from packet \\
        into specific \texttt{header} object
    };
    \end{visibleenv}
\end{tikzpicture}
\end{frame}

\begin{frame}[fragile]{P4: parsing}
\providecommand{\vemphA}[1]{\myemph<2>{#1}}
\providecommand{\vemphB}[1]{\myemph<3>{#1}}
\providecommand{\vemphC}[1]{\myemph<4>{#1}}
\providecommand{\vemphD}[1]{\myemph<5>{#1}}
\begin{Verbatim}[fontsize=\small,commandchars=\\@~]
parser MyParser(...) {
    ...
    state parse_ethernet {
        \vemphA@packet.extract(hdr.ethernet);~
        transition select(hdr.ethernet.etherType) {
            TYPE_IPV4: parse_ipv4;
            TYPE_IPV6: parse_ipv6
            default: accept;
        }
    }

    state parse_ipv4 {
        \vemphA@packet.extract(hdr.ipv4)~
        transition accept;
    }
    ....
}
\end{Verbatim}
\begin{tikzpicture}[overlay,remember picture]
    \begin{visibleenv}<2>
    \node[align=left,draw=red,fill=white,ultra thick] at ([yshift=-1cm,xshift=4cm]current page.center) {
        extract operations \\
        can be chained \\
        to extract multiple headers \\
        (for multiple layers \\
        or variable-length headers)
    };
    \end{visibleenv}
\end{tikzpicture}
\end{frame}

\begin{frame}[fragile]{P4: deparsing}
\providecommand{\vemphA}[1]{\myemph<2>{#1}}
\providecommand{\vemphB}[1]{\myemph<3>{#1}}
\providecommand{\vemphC}[1]{\myemph<4>{#1}}
\providecommand{\vemphD}[1]{\myemph<5>{#1}}
\begin{Verbatim}[fontsize=\small,commandchars=\\@~]
control MyDeparser(packet_out packet, in headers hdr) {
    apply {
        \vemphA@packet.emit~(hdr.ethernet);
        \vemphA@packet.emit~(hdr.ipv4);
        \vemphA@packet.emit~(hdr.ipv6);
    }
}
\end{Verbatim}
\begin{tikzpicture}[overlay,remember picture]
    \begin{visibleenv}<2>
    \node[align=left,draw=red,fill=white,ultra thick] at ([yshift=-2cm]current page.center) {
        use emit to put extracted bits back in packet \\
        ~  \\
        if header is not marked valid, emit does nothing \\
        (\texttt{extract(hdr.foo)} marks \texttt{hdr.foo} valid, \\
        can also use \texttt{hdr.foo.setValid()} and undo with \\
        \texttt{hdr.foo.setInvalid()})
    };
    \end{visibleenv}
\end{tikzpicture}
\end{frame}

\begin{frame}[fragile]{P4: ingress processing}
\providecommand{\vemphA}[1]{\myemph<2>{#1}}
\providecommand{\vemphB}[1]{\myemph<3>{#1}}
\providecommand{\vemphC}[1]{\myemph<4>{#1}}
\providecommand{\vemphD}[1]{\myemph<5>{#1}}
\begin{itemize}
\item example: send \textit{everything} to output port number 4
\end{itemize}
\begin{Verbatim}[fontsize=\small,commandchars=\\@~]
control MyIngress(...) {
    apply {
        \vemphA@standard_metadata.egress_spec = 4;~
    }
}
\end{Verbatim}
\begin{tikzpicture}[overlay,remember picture]
    \begin{visibleenv}<2>
    \node[align=left,draw=red,fill=white,ultra thick] at ([yshift=-2cm]current page.center) {
        standard\_metadata --- `hard-coded' phase reads \texttt{egress\_spec} \\
        ~ \\
        special \texttt{egress\_spec} value for ``discard frame''
    };
    \end{visibleenv}
\end{tikzpicture}
\end{frame}

\begin{frame}{switch decisions}
    \begin{itemize}
    \item most decisions switches/routers make are table lookups
    \vspace{.5cm}
    \item take field from packet (e.g. destination)
    \item lookup in table what to do with that
        (e.g. send to port X, throw error, etc.)
    \vspace{.5cm}
    \item P4 has special syntax for tables and table lookups
    \end{itemize}
\end{frame}

\begin{frame}[fragile]{P4: tables}
\providecommand{\vemphA}[1]{\myemph<2>{#1}}
\providecommand{\vemphB}[1]{\myemph<3>{#1}}
\providecommand{\vemphC}[1]{\myemph<4>{#1}}
\providecommand{\vemphD}[1]{\myemph<5>{#1}}
\begin{Verbatim}[fontsize=\small,commandchars=\\@~]
\vemphA@table~ mac_dst_exact {
    \vemphB@key~ = {
        \vemphB@hdr.ethernet.dstAddr~ : \vemphC@exact~;
    }
    \vemphB@actions~ = {
        drop,
        forward_to,
    }
    size = 1024;
    default_action = drop();
}
\end{Verbatim}
\begin{tikzpicture}[overlay,remember picture]
\tikzset{
    box/.style={
        align=left,
        draw=red,
        fill=white,
        ultra thick,
        anchor=west,
        at={([xshift=1cm]current page.center)},
    },
}
    \begin{visibleenv}<2>
    \node[box] {
        declare a table with a name \\
        done has part of ingress/outgress
    };
    \end{visibleenv}
    \begin{visibleenv}<3>
    \node[box] {
        key/value structure \\
        keys usually from headers \\
        values are actions to run
    };
    \end{visibleenv}
    \begin{visibleenv}<4>
    \node[box] {
        here, each table entry \\
        contains whole address \\
        P4 also supports \\
        partially matched keys
    };
    \end{visibleenv}
\end{tikzpicture}
\end{frame}

\begin{frame}{P4 table key types}
    \begin{itemize}
    \item exact --- full entry
    \item lpm --- longest prefix match
        \begin{itemize}
        \item table entries contain `prefixes' of header field(s)
        \item we'll see motivation for this when we talk about routing
        \item if multiple entries match, take longest one
        \end{itemize}
    \item ternary --- 0/1/don't care
        \begin{itemize}
        \item table entries contain key and mask
        \item entry matches if bits included in mask match
        \item other view: keys represented with 0/1/don't care `trits'
        \end{itemize}
    \end{itemize}
\end{frame}

\begin{frame}{content-addressable memory}
    \begin{itemize}
    \item high-end routers/switches have \textit{content addressable memory} (CAM)
    \item \ldots including \textit{ternary content addressable memory} (TCAM)
    \vspace{.5cm}
    \item = hardware that implements a table lookup
        \begin{itemize}
        \item `content' $\approx$ key being looked up
        \item looks kinda like highly associative cache
        \item probably lots of comparators
        \end{itemize}
    \item allows multigigabit frame processing speeds
    \item P4 goal: P4 programs compile to use CAM/TCAM when available
    \end{itemize}
\end{frame}

\begin{frame}{aside: TCAM cost}
    \begin{itemize}
    \item 2.7x more transistors than SRAM (common cache technology)
    \item probably much more power than SRAM
    \item hard to find recent quotes for price/power
    \item but probably 10s of dollars per megabyte
        \beign{itemize}
        \item e.g., secondary market price of Broadcom chip with 40Mbit of this is $\sim$\$180
        \item chip needs 80W heatsink, max 900MHz clock rate
        \item but chip has a bunch of other functionality, of course
        \end{itemize}
    \end{itemize}
\end{frame}

\begin{frame}[fragile]{P4: using a table}
\providecommand{\vemphA}[1]{\myemph<2>{#1}}
\providecommand{\vemphB}[1]{\myemph<3>{#1}}
\providecommand{\vemphC}[1]{\myemph<4>{#1}}
\providecommand{\vemphD}[1]{\myemph<5>{#1}}
\begin{Verbatim}[fontsize=\small,commandchars=\\@~]
control MyIngress(...) {
    ...
    table mac_dst_exact = { 
        /* seen earlier */ ...
    }
    apply {
        \vemphA@mac_dst_exact.apply();~
    }
}
\end{Verbatim}
\begin{tikzpicture}[overlay,remember picture]
\tikzset{
    box/.style={
        align=left,
        draw=red,
        fill=white,
        ultra thick,
        anchor=west,
        at={([xshift=1cm]current page.center)},
    },
}
    \begin{visibleenv}<2>
    \node[box] {
        apply operation invokes the action specified by the table
    };
    \end{visibleenv}
\end{tikzpicture}
\end{frame}

\begin{frame}[fragile]{P4: actions}
\providecommand{\vemphA}[1]{\myemph<2>{#1}}
\providecommand{\vemphB}[1]{\myemph<3>{#1}}
\providecommand{\vemphC}[1]{\myemph<4>{#1}}
\providecommand{\vemphD}[1]{\myemph<5>{#1}}
\begin{Verbatim}[fontsize=\small,commandchars=\\@~]
control MyIngress(...) {
    \vemphA@action drop()~ {
        mark_to_drop(\vemphB@standard_metadata~);
    }
    \vemphA@action forward_to(egressSpec_t port)~ {
        \vemphB@standard_metadata.egress_spec = port;~
    }
    \vemphA@action mark_and_forward(egressSpec_t port)~ {
        \vemphC@hdr.some_proto.value~ = 1;
        \vemphB@standard_metadata.egress_spec = port;~
    }
    ...
}
\end{Verbatim}
\begin{tikzpicture}[overlay,remember picture]
\tikzset{
    box/.style={
        align=left,
        draw=red,
        fill=white,
        ultra thick,
        anchor=north,
        at={([yshift=-1cm]current page.center)},
    },
}
    \begin{visibleenv}<2>
    \node[box] {
        actions basically function calls for packet \\
        can include parameters that are stored in table
    };
    \end{visibleenv}
    \begin{visibleenv}<3>
    \node[box] {
        typically, actions set standard metadata \\
        to indicate where to send frame next \\
        ~ \\
        actual sending/not sending frame done later
    };
    \end{visibleenv}
    \begin{visibleenv}<4>
    \node[box] {
        actions can also edit packet headers \\
        or do other table lookups \\
        (which we will need in the future) 
    };
    \end{visibleenv}
\end{tikzpicture}
\end{frame}

\begin{frame}[fragile]{P4: special actions}
    \begin{itemize}
    \item some ways switch can direct packet (incomplete list)
    \vspace{.5cm}
    \item send to the control plane
        \begin{itemize}
        \item special output port that goes to `general purpose' CPU
        \end{itemize}
    \item multicast/broadcast to multiple ports
        \begin{itemize}
        \item control plane can set \textit{multicast groups}
        \item makes multiple copies of packets
        \end{itemize}
    \end{itemize}
\end{frame}

