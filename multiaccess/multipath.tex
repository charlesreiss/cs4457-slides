
\begin{frame}{multipath interference}
\begin{tikzpicture}
\begin{pgfonlayer}{fg}
    \draw[fill=black] (0, 0) circle (.2cm);
    \draw[fill=black] (12, -4) circle (.2cm);
    \begin{scope}[ultra thick]
        \draw (-2, 2) rectangle (13, -5);
    \end{scope}
\end{pgfonlayer}
\tikzset{
    signal/.style={draw,violet,line width=.8mm,-Latex},
    signal fade/.style={draw,violet!60!white,line width=.8mm},
    signal fade2/.style={draw,violet!30!white,line width=.8mm},
    signal end/.style={-{Latex[sep=0.2cm]}},
}
\draw[signal] (0, 0) -- (12 / 3, 6 / 3);
\draw[signal fade,signal end] (12 / 3, 6 / 3) -- (12, -4);
\draw[signal] (0, 0) -- (-14 / 7, -4 / 7);
\draw[signal fade,signal end] (-14 / 7, -4 / 7) -- (12, -4);
\draw[signal] (0, 0) -- (14 * 13 / 14, -4 * 13 / 14);
\draw[signal fade, signal end] (14 * 13 / 14, -4 * 13 / 14) -- (12, -4);
\draw[signal] (0, 0) -- (14 * 5 / 6, -6 * 5 / 6);
\draw[signal fade, signal end] (14 * 5 / 6, -6 * 5 / 6) -- (12, -4);

\draw[signal,signal end] (0, 0) -- (12, -4);
\end{tikzpicture}
\end{frame}

\begin{frame}{simple self-interference}
\begin{tikzpicture}
\begin{axis}[name=top1,hide axis,width=7cm,height=4cm,ymin=-1,ymax=1]
    \addplot[domain=0:1000,samples=1000] {sin(x*4) * sin(0.75+x/3)};
    \coordinate (top1 tl) at (axis cs:0, 1);
    \coordinate (top1 br) at (axis cs:1000, -1);
\end{axis}
\coordinate (middle base) at ([yshift=-2cm]top1.east);
\coordinate (low base) at ([yshift=-4cm]top1.east);
\begin{axis}[anchor=east,name=middle1,at={(middle base)},width=7cm,height=4cm,hide axis,
    ymin=-1,ymax=1]
    \addplot[domain=0:1000,samples=1000] {0.4*(sin((x+45)*4) * sin(0.75+(x + 45)/3))};
    \coordinate (middle1 tl) at (axis cs:0, 0.4);
    \coordinate (middle1 br) at (axis cs:1000, -0.4);
\end{axis}
\begin{axis}[hide axis,width=7cm,height=4cm,anchor=east,name=combine1,at={(low base)},
    ymin=-1,ymax=1]
    \addplot[domain=0:1000,samples=1000] {
     sin(x*4) * sin(0.75+x/3) +
     0.5*(sin((x+45)*4) * sin(0.75+(x + 45)/3))
    };
\end{axis}
\coordinate (top2 base) at ([xshift=2cm]top1.east);
\begin{axis}[anchor=west,at={(top2 base)},name=top2,hide axis,width=7cm,height=4cm,ymin=-1,ymax=1]
    \addplot[domain=0:1000,samples=1000] {sin(x*4) * sin(0.75+x/3)};
    \coordinate (top2 tl) at (axis cs:0, 1);
    \coordinate (top2 br) at (axis cs:1000, -1);
\end{axis}
\coordinate (middle base2) at ([yshift=-2cm]top2.east);
\coordinate (low base2) at ([yshift=-4cm]top2.east);
\begin{axis}[anchor=east,name=middle2,at={(middle base2)},width=7cm,height=4cm,hide axis,
    ymin=-1,ymax=1]
    \addplot[domain=0:1000,samples=1000] {0.4*(sin((x+25)*4) * sin(0.75+(x + 25)/3))};
    \coordinate (middle2 tl) at (axis cs:0, 0.4);
    \coordinate (middle2 br) at (axis cs:1000, -0.4);
\end{axis}
\begin{axis}[hide axis,width=7cm,height=4cm,anchor=east,name=combine2,at={(low base2)},
    ymin=-1,ymax=1]
    \addplot[domain=0:1000,samples=1000] {
     sin(x*4) * sin(0.75+x/3) +
     0.5*(sin((x+25)*4) * sin(0.75+(x + 25)/3))
    };
\end{axis}
\node[font=\large]  at (middle1.west) {+};
\node[font=\large]  at (middle2.west) {+};
\draw[ultra thick] (middle1.south west) -- (middle1.south east);
\draw[ultra thick] (middle2.south west) -- (middle2.south east);
\begin{visibleenv}<2>
\draw[very thick,red] (top1 tl) rectangle (top1 br);
\draw[very thick,red] (top2 tl) rectangle (top2 br);
\node[font=\small,anchor=west,red] at (top1.east) {original};
\draw[very thick,blue] (middle1 tl) rectangle (middle1 br);
\draw[very thick,blue] (middle2 tl) rectangle (middle2 br);
\node[font=\small,anchor=west,blue] at (middle1.east) {reflected};
\end{visibleenv}
\end{tikzpicture}
\end{frame}

\begin{frame}{interference and wavelengths}
    \begin{itemize}
    \item radio wave frequency determines wavelength
        \begin{itemize}
        \item wifi: centimeters
        \item distance between `peaks' in power
        \item half wavelength = distance between peak and trough
        \end{itemize}
    \item if distances make peaks align with troughs, lower power
    \item if distances make peaks align with peaks, higher power
    \end{itemize}
\end{frame}

\begin{frame}{interference and symbols}
    \begin{itemize}
    \item separate from wavelength: how long are `symbols'
        \begin{itemize}
        \item the things we turn into bits
        \end{itemize}
    \item for long paths/high symbol rate, might receive delayed signal overlaid on non-delayed
    \end{itemize}
\end{frame}


\begin{frame}{multipath interference}
    \begin{itemize}
    \item quite complicated to predict in three dimensions, variable reflectivity, \ldots
    \item varies on frequency
        \begin{itemize}
        \item switching to close different channel may change a lot
        \end{itemize}
    \item can help and hurt signal reception
    \item especially bad when `inter-symbol interference'
        \begin{itemize}
        \item can be mitigated some by choice of message encoding
        \end{itemize}
    \vspace{.5cm}
    \item contributes to irregular `dead zones', etc.
    \end{itemize}
\end{frame}

\begin{frame}{deliberate multipath / MIMO}
    \begin{itemize}
    \item can have:
    \item multiple sending antennas (inputs) (MI)
    \item multiple receiving antennas (outputs) (MO)
    \item some simple ideas:
    \vspace{.5cm}
    \item receive signal twice, hope one antenna not in dead zone
    \item send signal twice, have receiver tell you which is better
    \end{itemize}
\end{frame}

\begin{frame}{multiple spatial streams}
\begin{tikzpicture}
\begin{scope}[every path/.style={fill=black!50,draw=black!50,text=black}]
    \path (-.5, .5) rectangle ++(.5, -4) node[midway] {A};
    \path (12, .5) -- (12.5, 0.3) -- (11.5, -3.5) -- (11, -3.3) -- cycle;
    \node at (11.7, -1.5) {B};
    \path (0, 0) circle (2mm) coordinate (A1) node[above right=1mm]{$A_1$};
    \path (0, -3) circle (2mm) coordinate (A2) node[below right=1mm]{$A_2$};
    \path (11.7, 0) circle (2mm) coordinate (B1) node[above left=1mm]{$B_1$};
    \path (11, -3) circle (2mm) coordinate (B2) node[below left=1mm]{$B_2$};
\end{scope}
\tikzset{
    strong/.style={line width=1.2mm,arrows={{Butt Cap[sep=5mm]}-{Latex[sep=5mm]}}},
    stronger/.style={line width=1.3mm,arrows={{Butt Cap[sep=5mm]}-{Latex[sep=5mm]}}},
    medium/.style={line width=1mm,arrows={{Butt Cap[sep=5mm]}-{Latex[sep=5mm]}}},
    weak/.style={line width=0.8mm,arrows={{Butt Cap[sep=5mm]}-{Latex[sep=5mm]}}},
}
\foreach \x/\y/\lw in {A1/B1/strong,A2/B2/stronger,A1/B2/medium,A2/B1/weak} {
    \draw[\lw] (\x) -- (\y);
}
\end{tikzpicture}
\begin{itemize}
\item<2-> signals might mix based on relative strengths, say
\begin{align*}
B_1 &\sim& 0.9A_1 &+& 0.7A_2 \\
B_2 &\sim& A_1 &+& 1.2A_2 
\end{align*}
\end{itemize}
\end{frame}

\begin{frame}{simplified multipath model}
\begin{align*}
B_1 &\sim& 0.9A_1 &+& 0.7A_2 \\
B_2 &\sim& A_1 &+& 1.2A_2 
\end{align*}
\begin{itemize}
\item if we can estimate coefficients\ldots
\item B can solve for $A_1$, $A_2$
\vspace{.5cm}
\item A can send with pattern to help estimate coefficients:
    \begin{itemize}
    \item example: send full power from $A_1$ ($(B_1,B_2)=(0.9, 1)$), then from $A_2$ ($(B_1,B_2)=(0.7, 1.2)$)
    \end{itemize}
\end{itemize}
\end{frame}

\begin{frame}{correlation problem}
    \begin{itemize}
    \item when two `spatial streams' correlated, harder to solve for them
    \item example: suppose $(B_1,B_2)=(A_1+A_2,A_1+1.1A_2)$
    \vspace{.5cm}
    \item $A_2=10B_1-10B_2$
    \item need 10x precision in $B_1$, $B_2$ to get 1x precision in $A_2$
    \end{itemize}
\end{frame}
