\usetikzlibrary{calc}

\begin{frame}{free space wireless transmission}
\begin{itemize}
\item assuming 2.4GHz Wifi, no obstacles, omnidirectional transmission
\end{itemize}
\begin{tikzpicture}
\begin{axis}[
    xlabel={distance from transmitter (m)},
    ylabel={lost signal (dB)},
    width=12cm,
    height=5.5cm,
    xmode=log,
]
\addplot table {../multiaccess/freespacemodel.table};
%\addplot[domain=0.1:100,samples=100] { 20*log(d/1000.0)/log(10) + 20*log(2.4)/log(10) + 92.45 };
\end{axis}
\end{tikzpicture}
\end{frame}

\begin{frame}\frametitle{modifying free space}
    \begin{itemize}
    \item above model assumes no directionality to antennas
        \begin{itemize}
        \item signal going twice as far = needs to spread out $2^3$ times as much (3D)
        \end{itemize}
    \item can do better with directional signal
        \begin{itemize}
        \item wifi distance records use this strategy + other changes
        \end{itemize}
    \end{itemize}
\end{frame}

\begin{frame}{signal to noise ratio}
    \begin{itemize}
    \item often measure power of signal versus power of `noise'
        \begin{itemize}
        \item simple model: noise = `noise floor' + interfering transmission
        \end{itemize}
    \item how strong the `symbols' (for framing protocol) are relative to everything else
    \item more signal to noise = more bits per unit time
        \begin{itemize}
        \item signal to noise limits maximum achievable regardless of how clever framing is
        \end{itemize}
    \end{itemize}
\end{frame}

\begin{frame}\frametitle{free space example}
    \begin{itemize}
    \item 100 mW signal (= 20 dBm)
    \item at 10 metres for 2.4GHz, about 60 dB of loss $\sim 10^-4$ mW of power in antenna
    \vspace{.5cm}
    \item suppose `noise floor' is 80 dBm = $10^{-8}$ mW
    \item signal to noise ratio = $10^{-4}/10^{-8} = 10^{4} = 10000 = 40 \text{dB}$
    \end{itemize}
\end{frame}

\begin{frame}
\frametitle{SNR and bitrate}
    \begin{itemize}
    \item theory (Shannon-Hartley): max bitrate = $B\log_2\left(1+\frac{S}{N}\right)$
    \item $B$ = bandwidth of channel
    \item $S$, $N$ = power per bandwidth of signal, noise (non-log scale)
    \end{itemize}
\end{frame}

\begin{frame}
\frametitle{SNR and bitrate example}
    \begin{itemize}
    \item max bitrate = $B\log_2\left(1+\frac{S}{N}\right)$
    \item $B$ = bandwidth = 40MHz maximum for 802.11n in the 2.4Ghz band
    \item $S$, $N$ = power per bandwidth of signal, noise (non-log scale)
    \item previous computation $\frac{S}{N} = 10000$ for 10 metre distance
    \item $\rightarrow 40MHz\log_2\left(1 + \frac{10000}{1}\right) = \log_2(10001)\text{Mbit/s} \approx 531 \text{Mbit/sec}$
    \end{itemize}
\end{frame}

\begin{frame}{signal to noise ratio and bitrate}
    \begin{itemize}
    \item can recover from more intereference with lower bitrate
    \vspace{.5cm}
    \item example: 802.11b supports 1, 2, 5.5, 11 Mbit/sec
    \item most devices measure signal-to-noise ratio and set bitrate
    \end{itemize}
\end{frame}

\begin{frame}
\includegraphics[height=0.9\textheight]{../multiaccess/haratcherev-fig1.png}
\imagecredit{Haratcherev, Langendoen, Lagendijk, Sips, ``Hybrid Rate Control for IEEE 802.11'' (MobiWac'04), Figure 1}
\end{frame}

\begin{frame}
\frametitle{rate control}
\begin{itemize}
    \item not always best to transmit at fastest rate
    \item $\rightarrow$ rate control algorithms
    \item typical method:
        \begin{itemize}
        \item have menu of encoding schemes, with different rates, SnR
        \item measure succes rate and/or signal to noise rate
        \end{itemize}
\end{itemize}
\end{frame}


\begin{frame}{receiving multiple nodes}
\begin{tikzpicture}
\begin{pgfonlayer}{fg}
    \draw[fill=black] (0, 0) circle (.2cm)
        node[above=1mm] {D};
    \draw[fill=black] (2, -1) circle (.2cm)
        node[below=1mm] {A};
    \draw[fill=black] (12, 1) circle (.2cm)
        node[below=1mm] {B};
\end{pgfonlayer}
\tikzset{
    signal/.style={draw,violet,line width=1.4mm,-Latex},
    signal fade/.style={draw,violet!60!white,line width=.8mm},
    signal fade2/.style={draw,violet!30!white,line width=.4mm},
    signal end/.style={-{Latex[sep=0.2cm]}},
    signalB/.style={draw,blue,line width=1.4mm,-Latex},
    signal fadeB/.style={draw,blue!60!white,line width=.8mm},
    signal fade2B/.style={draw,blue!30!white,line width=.4mm},
}
\draw[signal,signal end] (2, -1) -- (0, 0);
\draw[signalB,-] (12, 1) -- (10, 10/12);
\draw[signal fadeB,-] (10, 10/12) -- (8, 8/12);
\draw[signal fade2B,signal end] (8, 8/12) -- (0, 0);
\end{tikzpicture}
\begin{itemize}
\item A's signal much much stronger at D
\item D may still receive A's signal\ldots
\item because B relatively too weak to interfere
\end{itemize}
\end{frame}

\begin{frame}{collision (non-)detection}
\begin{tikzpicture}
\begin{pgfonlayer}{fg}
    \draw[fill=black] (7, 0) circle (.2cm)
        node[above=1mm] {D};
    \draw[fill=black] (2, 1) circle (.2cm)
        node[below=1mm] {A};
    \draw[fill=black] (12, 1) circle (.2cm)
        node[below=1mm] {B};
\end{pgfonlayer}
\tikzset{
    signal/.style={draw,violet,line width=1.4mm,-Latex},
    signal fade/.style={draw,violet!60!white,line width=.8mm},
    signal fade2/.style={draw,violet!30!white,line width=.4mm},
    signal end/.style={-{Latex[sep=0.2cm]}},
    signalB/.style={draw,blue,line width=1.4mm,-Latex},
    signal fadeB/.style={draw,blue!60!white,line width=.8mm},
    signal fade2B/.style={draw,blue!30!white,line width=.4mm},
}
\draw[signal,-] (2, 1) -- (4, 3/5);
\draw[signal fade,signal end] (4, 3/5) -- (7, 0);
\draw[signalB,-] (12, 1) -- (10, 3/5);
\draw[signal fadeB,signal end] (10, 3/5) -- (7, 0);'
\begin{visibleenv}<1>
    \node[anchor=north,draw=red,align=left,very thick] at (7, -0.2) {
        D receives junk
    };
\end{visibleenv}
\begin{visibleenv}<2>
\draw[signal,-] (2, 1) -- (4.5, 1);
\draw[signal fade,-] (4.5, 1) -- (7, 1);
\draw[signal fade2,signal end] (7, 1) -- (12, 1);
\draw[red,very thick] (12, 1) circle (.3cm);
    \node[anchor=north,draw=red,align=left,very thick] at (12, 0.2) {
        B's own transmission\\
        is much stronger than A's \\
        cannot detect collision itself
    };
\end{visibleenv}
\end{tikzpicture}
\end{frame}

\begin{frame}{carrier-sense}
\begin{tikzpicture}
\begin{pgfonlayer}{fg}
    \draw[fill=black] (7, 0) circle (.2cm)
        node[above=1mm] {D};
    \draw[fill=black] (2, 1) circle (.2cm)
        node[below=1mm] {A};
    \draw[fill=black] (12, 1) circle (.2cm)
        node[below=1mm] {B};
\end{pgfonlayer}
\tikzset{
    signal/.style={draw,violet,line width=1.4mm,-Latex},
    signal fade/.style={draw,violet!60!white,line width=.8mm},
    signal fade2/.style={draw,violet!30!white,line width=.4mm},
    signal end/.style={-{Latex[sep=0.2cm]}},
}
\draw[signal,-] (2, 1) -- (4, 3/5);
\draw[signal fade,signal end] (4, 3/5) -- (7, 0);
\draw[signal,-] (2, 1) -- (4.5, 1);
\draw[signal fade,-] (4.5, 1) -- (7, 1);
\draw[signal fade2,signal end] (7, 1) -- (12, 1);
\draw[red,very thick] (12, 1) circle (.3cm);
    \node[anchor=north,draw=red,align=left,very thick] at (12, 0.2) {
        B can detect A's transmitting \\
        avoid starting transmission now
    };
\end{tikzpicture}
\end{frame}

\begin{frame}{carrier-not-sensing}
\begin{tikzpicture}
\begin{pgfonlayer}{fg}
    \draw[fill=black] (7, 0) circle (.2cm)
        node[above=1mm] {D};
    \draw[fill=black] (2, 1) circle (.2cm)
        node[below=1mm] {A};
    \draw[fill=black] (14, 1) circle (.2cm)
        node[below=1mm] {B};
\end{pgfonlayer}
\tikzset{
    signal/.style={draw,violet,line width=1.4mm,-Latex},
    signal fade/.style={draw,violet!60!white,line width=.8mm},
    signal fade2/.style={draw,violet!30!white,line width=.4mm},
    signal fade3/.style={draw,violet!15!white,line width=.2mm},
    signal end/.style={-{Latex[sep=0.2cm]}},
    range mark/.style={draw,violet!50!black,dotted,line width=1mm},
}
\begin{scope}
    \clip (1, -6) rectangle (15, 2);
    \draw[overlay,range mark,fill=violet!10] (2, 1) circle (10.5 cm);
        \node[font=\Huge] at (4, -2) {A's range};
\end{scope}
\draw[signal,-] (2, 1) -- (4, 3/5);
\draw[signal fade,signal end] (4, 3/5) -- (7, 0);
\draw[signal,-] (2, 1) -- (4.5, 1);
\draw[signal fade,-] (4.5, 1) -- (7, 1);
\draw[signal fade2,-] (7, 1) -- (9, 1);
\draw[signal fade3,signal end] (9, 1) -- (12.5, 1);
\draw[red,very thick] (14, 1) circle (.3cm);
    \node[anchor=north,draw=red,align=left,very thick,fill=white] at (14, 0.2) {
        B doesn't know \\ A is transmitting \\
        because A's signal\\
        is too weak
    };
\end{tikzpicture}
\end{frame}

\begin{frame}{`hidden node' problem}
\begin{tikzpicture}
\begin{pgfonlayer}{fg}
    \draw[fill=black] (4, 0) circle (.2cm)
        node[above=1mm] {D};
    \draw[fill=black] (2, 1) circle (.2cm)
        node[below=1mm] {A};
    \draw[fill=black] (6, 0.5) circle (.2cm)
        node[below=1mm] {B};
\end{pgfonlayer}
\tikzset{
    signal/.style={draw,violet,line width=1.4mm,-Latex},
    signalB/.style={draw,blue,line width=1.4mm,-Latex},
    signal fade/.style={draw,violet!60!white,line width=.8mm},
    signal fade2/.style={draw,violet!30!white,line width=.4mm},
    signal fade3/.style={draw,violet!15!white,line width=.2mm},
    signal end/.style={-{Latex[sep=0.2cm]}},
    range mark/.style={draw,violet!50!black,dotted,line width=1mm},
    range markB/.style={draw,blue!50!black,dashed,line width=1mm},
}
\draw[signal,signal end] (2, 1) -- (4,0);
\draw[signalB,signal end] (6, .5) -- (4,0);
\begin{scope}
    \clip (-1, -6) rectangle (15, 2);
    \draw[overlay,range mark] (2, 1) circle (3.5 cm);
    \draw[overlay,range markB] (6, 0.5) circle (3.5 cm);
\end{scope}
\end{tikzpicture}
\begin{itemize}
\item B is a `hidden node' for A and vice-versa
\end{itemize}
\end{frame}

\begin{frame}{carrier false sensing}
\begin{tikzpicture}
\begin{pgfonlayer}{fg}
    \draw[fill=black] (4, 1) circle (.2cm)
        node[below=1mm] {A};
    \draw[fill=black] (1, -1) circle (.2cm)
        node[below=1mm] {C};
    \draw[fill=black] (9, 1) circle (.2cm)
        node[below=1mm] {B};
    \draw[fill=black] (12, 1) circle (.2cm)
        node[below=1mm] {D};
\end{pgfonlayer}
\tikzset{
    signal/.style={draw,violet,line width=1.4mm,-Latex},
    signal fade/.style={draw,violet!60!white,line width=.8mm},
    signal fade2/.style={draw,violet!30!white,line width=.4mm},
    signal fade3/.style={draw,violet!15!white,line width=.2mm},
    signal end/.style={-{Latex[sep=0.2cm]}},
    signalB/.style={draw,blue,line width=1.4mm,-Latex},
    signal fadeB/.style={draw,blue!60!white,line width=.8mm},
    signal fade2B/.style={draw,blue!30!white,line width=.4mm},
    range mark/.style={draw,violet!50!black,dotted,line width=1mm},
    range markB/.style={draw,blue!50!black,dashed,line width=1mm},
}
\draw[signal,signal end] (4, 1) -- (1, -1);
\begin{visibleenv}<1>
\draw[signalB,signal end] (9, 1) -- (12, 1);
\end{visibleenv}
\begin{visibleenv}<2>
\draw[signal fade,signal end] (4, 1) -- (9, 1);
\end{visibleenv}
\begin{scope}
    \clip (1, -6) rectangle (15, 2);
    \draw[overlay,range mark] (4, 1) circle (6cm);
    \draw[overlay,range markB] (9, 1) circle (6cm);
\end{scope}
\begin{visibleenv}<1>
    \node[align=left,anchor=north,red,very thick] at (6, -.5) {
        A$\rightarrow$C and B$\rightarrow$D won't collide! \\
    };
\end{visibleenv}
\begin{visibleenv}<2>
    \node[align=left,anchor=north,red,very thick,fill=white,draw=red] at (6, -.5) {
        B will detect carrier \\
        may avoid transmitting to D \\
        even though there is no interference!
    };
\end{visibleenv}
\end{tikzpicture}
\end{frame}


\begin{frame}{hidden/exposed node summary}
\begin{tikzpicture}
\begin{scope}[x=0.3cm,y=0.5cm]
    \begin{pgfonlayer}{fg}
        \draw[fill=black] (4, 1) circle (.2cm)
            node[below=1mm] {A};
        \draw[fill=black] (1, -1) circle (.2cm)
            node[below=1mm] {C};
        \draw[fill=black] (9, 1) circle (.2cm)
            node[below=1mm] {B};
        \draw[fill=black] (12, 1) circle (.2cm)
            node[below=1mm] {D};
    \end{pgfonlayer}
    \tikzset{
        signal/.style={draw,violet,line width=1.4mm,-Latex},
        signal fade/.style={draw,violet!60!white,line width=.8mm},
        signal fade2/.style={draw,violet!30!white,line width=.4mm},
        signal fade3/.style={draw,violet!15!white,line width=.2mm},
        signal end/.style={-{Latex[sep=0.2cm]}},
        signalB/.style={draw,blue,line width=1.4mm,-Latex},
        signal fadeB/.style={draw,blue!60!white,line width=.8mm},
        signal fade2B/.style={draw,blue!30!white,line width=.4mm},
        range mark/.style={draw,violet!50!black,dotted,line width=.4mm},
        range markB/.style={draw,blue!50!black,dashed,line width=.4mm},
    }
    \draw[violet,dotted,very thick,signal end] (4, 1) -- (1, -1);
    \draw[blue,dotted,very thick,signal end] (9, 1) -- (12, 1);
    \begin{scope}
        \clip (-5, -5) rectangle (20, 4);
        \draw[overlay,range mark] (4, 1) circle (2cm);
        \draw[overlay,range markB] (9, 1) circle (2cm);
    \end{scope}
    \node[anchor=west,align=left,font=\small] at (18, 0) {
        \textit{exposed node problem}: A and B \\
        detect each other's transmissions \\
        but don't interfere 
    };
\end{scope}
\begin{scope}[yshift=-4cm,x=0.3cm,y=0.5cm]
    \begin{pgfonlayer}{fg}
        \draw[fill=black] (1, 1) circle (.2cm)
            node[below=1mm] {A};
        \draw[fill=black] (4, -1) circle (.2cm)
            node[below=1mm] {C};
        \draw[fill=black] (8, -1) circle (.2cm)
            node[below=1mm] {B};
    \end{pgfonlayer}
    \tikzset{
        signal/.style={draw,violet,line width=1.4mm,-Latex},
        signal fade/.style={draw,violet!60!white,line width=.8mm},
        signal fade2/.style={draw,violet!30!white,line width=.4mm},
        signal fade3/.style={draw,violet!15!white,line width=.2mm},
        signal end/.style={-{Latex[sep=0.2cm]}},
        signalB/.style={draw,blue,line width=1.4mm,-Latex},
        signal fadeB/.style={draw,blue!60!white,line width=.8mm},
        signal fade2B/.style={draw,blue!30!white,line width=.4mm},
        range mark/.style={draw,violet!50!black,dotted,line width=.4mm},
        range markB/.style={draw,blue!50!black,dashed,line width=.4mm},
    }
    \begin{scope}
        \clip (-1, -5) rectangle (15, 4);
        \draw[overlay,range mark] (0, 1) circle (2cm);
        \draw[overlay,range markB] (8, -1) circle (2cm);
    \end{scope}
    \draw[violet,dotted,very thick,signal end] (1, 1) -- (4, -1);
    \draw[blue,dotted,very thick,signal end] (8, -1) -- (4, -1);
    \node[anchor=west,align=left,font=\small] at (18, 0) {
        \textit{hidden node problem}: A and B interfere \\
        when sending to C\\
        but can't detect each other's transmissions
    };
\end{scope}
\end{tikzpicture}
\end{frame}
