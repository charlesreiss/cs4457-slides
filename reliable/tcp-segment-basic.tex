\usetikzlibrary{patterns}
\begin{frame}[fragile]{TCP segment format}
\begin{tikzpicture}
    \tikzset{
        box/.style={draw,thick},
        box unused/.style={draw,thick,pattern=north west lines},
        box label/.style={midway,font=\small,align=center},
        box label flags/.style={midway,font=\fontsize{8}{9}\selectfont,align=center},
        hi on/.style={alt=<#1>{ultra thick,fill=red!10}},
        explain box 1/.style={draw=red,line width=0.8mm,fill=white,anchor=center,at={(explain box loc 1)},align=center},
        explain box 2/.style={draw=red,line width=0.8mm,fill=white,anchor=center,at={(explain box loc 2)},align=center},
        explain box 3/.style={draw=red,line width=0.8mm,fill=white,anchor=center,at={(explain box loc 3)},align=center},
    }
    \begin{scope}[x=4.7mm,y=9mm]
        \coordinate (explain box loc 1) at (16, -3.1);
        \coordinate (explain box loc 2) at (16, -5.1);
        \coordinate (explain box loc 3) at (16, -1.1);
        \path[shading=axis,top color=white,bottom color=black!20] (0, 1) rectangle (32, 0)
            node[box label] {(lower layer header)};
        \path[box,hi on=2] (0, 0) rectangle (16, -1) node[box label] {source port (16b)};
        \path[box,hi on=2] (16, 0) rectangle (32, -1) node[box label] {destination port (16b)};
        \path[box,hi on=3] (0, -1) rectangle (32, -2) node[box label] {sequence number (32b)};
        \path[box,hi on=4] (0, -2) rectangle (32, -3) node[box label] {acknowledgment number (32b)};
        \path[box,hi on=10] (0, -3) rectangle (4, -4) node[box label] {data offset\\(4b)};
        \path[box unused] (4, -3) rectangle (8, -4);
        \path[box,hi on=9] (8, -3) rectangle ++(1, -1) node[box label flags]{C\\W\\R};
        \path[box,hi on=9] (9, -3) rectangle ++(1, -1) node[box label flags]{E\\C\\E};
        \path[box,hi on=6] (10, -3) rectangle ++(1, -1) node[box label flags]{U\\R\\G};
        \path[box,hi on=4] (11, -3) rectangle ++(1, -1) node[box label flags]{A\\C\\K};
        \path[box,hi on=7] (12, -3) rectangle ++(1, -1) node[box label flags]{P\\S\\H};
        \path[box,hi on=8] (13, -3) rectangle ++(1, -1) node[box label flags]{R\\S\\T};
        \path[box,hi on=8] (14, -3) rectangle ++(1, -1) node[box label flags]{S\\Y\\N};
        \path[box,hi on=8] (15, -3) rectangle ++(1, -1) node[box label flags]{F\\I\\N};
        \path[box,hi on=5] (16, -3) rectangle ++(16, -1) node[box label] {window size (16b)};
        \path[box] (0, -4) rectangle ++(16, -1) node[box label] {checksum (16b)};
        \path[box, hi on=6] (16, -4) rectangle ++(16, -1) node[box label] {urgent pointer (16b)};
        \path[box,hi on=10] (0, -5) rectangle ++(32, -1) node[box label] {options (variable)};
        \draw[line width=1mm,white] (-.2, -5.55) -- (.2, -5.45);
        \draw[line width=1mm,white] (32 -.2, -5.55) -- (32 + .2, -5.45);
        \draw[very thick,black] (-.2, -5.5) -- (.2, -5.4);
        \draw[very thick,black] (32 -.2, -5.5) -- (32 + .2, -5.4);
        \draw[very thick,black] (-.2, -5.6) -- (.2, -5.5);
        \draw[very thick,black] (32 -.2, -5.6) -- (32 + .2, -5.5);
        \path[overlay,shading=axis,top color=black!20,bottom color=white] (0, -6) rectangle (32, -7)
            node[box label] {(data)};
    \end{scope}
    \begin{visibleenv}<2>
        \node[explain box 1] {
            ports identify program/socket on machines \\
            (we'll talk more when we cover sockets) \\
            ~ \\
            machines identified in lower layer headers 
        };
    \end{visibleenv}
    \begin{visibleenv}<3>
        \node[explain box 2] {
            byte number for first byte of data in this packet
        };
    \end{visibleenv}
    \begin{visibleenv}<4>
        \node[explain box 2] {
            ack number = 1 + byte number of largest byte acknowledged \\
            only meaningful if ACK `flag' is 1
        };
    \end{visibleenv}
    \begin{visibleenv}<5>
        \node[explain box 3] {
            window size is receive window size \\
            tells sender how much receiver will accept \\
            sender window could/often will be different \\
            (and not directly visible in packets)
        };
    \end{visibleenv}
    \begin{visibleenv}<6>
        \node[explain box 3] {
            some fields related to `urgent data' mechanism \\
            almost never used today
        };
    \end{visibleenv}
    \begin{visibleenv}<7>
        \node[explain box 3] {
            PSH (push) `flag' is hint that sender does not have \\
            more data to send right away
        };
    \end{visibleenv}
    \begin{visibleenv}<8>
        \node[explain box 3] {
            RST (reset), SYN (synchornize), FIN flags \\
            used for connnection management \\
            (we'll talk more when we cover sockets)
        };
    \end{visibleenv}
    \begin{visibleenv}<9>
        \node[explain box 3] {
            CWR (congestion window reduced) and \\
            ECE (explicit congestion notification echo) flags \\
            sometimes used as part of setting window size \\
            to match network conditions (later topic for us)
        };
    \end{visibleenv}
    \begin{visibleenv}<10>
        \node[explain box 3] {
            header can have variable number of ``options'' \\
            technically optional, almost always used today \\
            ~ \\
            size of header indicated by \textit{data offset} \\
            (data offset is units of 32-bit words, not bytes)
        };
    \end{visibleenv}
\end{tikzpicture}
\end{frame}

\begin{frame}\frametitle{exercise: maximum throughput}
    \begin{itemize}
    \item let's say we have a receiver window size of 65535 bytes
    \item and a round-trip time of 100 ms
    \vspace{.5cm}
    \item if we want to avoid sending data the receiver will reject as outside
    its window, maximum throughput?
    \end{itemize}
\begin{tabular}{ll}
A. around 32kbyte/sec &
B. around 64kbyte/sec \\
C. around 128kbyte/sec &
D. around 320kbyte/sec \\
E. around 640kbyte/sec &
F. around 1280kbyte/sec \\
G. something else \\
\end{tabular}
\end{frame}

\begin{frame}{selected TCP options}
    \begin{itemize}
    \item window size scale factor
        \begin{itemize}
        \item allow receiver window sizes greater than 64k
        \item needed to get reasonable bandwidth on modern networks
        \end{itemize}
    \item timestamps
        \begin{itemize}
        \item allow figuring out round trip time to estimate timeout
        \item extend 32-bit sequence number, which is too small for multi-gigabit networks
        \end{itemize}
    \item selective acknowledgements
        \begin{itemize}
        \item allow providing information about `holes' in received data
        \item example: I got bytes 1--5000, 6000--7000, 8000--9000
        \item without it would only say 5000
        \end{itemize}
    \end{itemize}
\end{frame}

\begin{frame}[fragile]{selected TCP option formats}
\begin{tikzpicture}
    \tikzset{
        box/.style={draw,thick},
        box unused/.style={draw,thick,pattern=north west lines},
        box label/.style={midway,font=\small,align=center},
        box label flags/.style={midway,font=\fontsize{8}{9}\selectfont,align=center},
        hi on/.style={alt=<#1>{ultra thick,fill=red!10}},
        explain box 1/.style={draw=red,line width=0.8mm,fill=white,anchor=center,at={(explain box loc 1)},align=center},
        explain box 2/.style={draw=red,line width=0.8mm,fill=white,anchor=center,at={(explain box loc 2)},align=center},
        explain box 3/.style={draw=red,line width=0.8mm,fill=white,anchor=center,at={(explain box loc 3)},align=center},
    }
    \begin{scope}[x=4.7mm,y=9mm]
        \coordinate (explain box loc 1) at (16, -5.1);
        \coordinate (explain box loc 2) at (16, -6.1);
        \coordinate (explain box loc 3) at (16, -1.1);
        \path[box,hi on=3] (0, 0) rectangle (8, -1) node[box label] {kind (8b)};
        \path[box,hi on=2] (8, 0) rectangle (16, -1) node[box label] {length (8b)};
        \path[shading=axis,left color=black!20,right color=white] (16, 0) rectangle (32, -1) node[box label] {option data};

        \node[anchor=south] at (16, -2) {\myemph<4>{window scale option}:};
        \path[box,hi on=4,hi on=3] (0, -2) rectangle ++(8, -1) node[box label] {\tt 3};
        \path[box,hi on=2,hi on=4] (8, -2) rectangle ++(8, -1) node[box label] {\tt 3};
        \path[box,hi on=4] (16, -2) rectangle ++(8, -1) node[box label] {shift count};
        
        \node[anchor=south] at (16, -4) {\myemph<5>{timestamps}:};
        \path[box,hi on=3] (0, -4) rectangle ++(8, -1) node[box label] {\tt 8};
        \path[box,hi on=2] (8, -4) rectangle ++(8, -1) node[box label] {\tt 10};
        \path[box] (16, -4) rectangle ++(16, -1) node[box label] {sender TS};
        \path[box,hi on=6] (0, -5) rectangle ++(16, -1) node[box label] {echoed TS};
    \end{scope}
    \begin{visibleenv}<2>
        \node[explain box 2] {
            length field permits skipping unrecognized options
        };
    \end{visibleenv}
    \begin{visibleenv}<3>
        \node[explain box 2] {
            unique kind codes for each option \\
            list of valid codes maintained by IANA \\
            (Internet Assigned Numbers Authority) 
        };
    \end{visibleenv}
    \begin{visibleenv}<4>
        \node[explain box 1] {
            sent only in connection setup \\
            only takes effect if both sides send it \\
            sets amount to left-shift \\
            all window size fields by
        };
    \end{visibleenv}
    \begin{visibleenv}<6>
        \node[explain box 3] {
            echoed timestamp --- \\
            only valid in ACK messages \\
            copy of timestamp sent in message ACK'd \\
            allows computing round-trip-time
        };
    \end{visibleenv}
\end{tikzpicture}
\end{frame}

\begin{frame}\frametitle{TCP connection setup preview}
\begin{itemize}
\item client$\rightarrow$server: SYN flag + seq=randomA [no data]
\item server$\rightarrow$client: SYN+ACK flags  + seq=randomB ack=randomA+1 [no data]
\item client$\rightarrow$server: ACK flags  + seq=randomA+1 ack=randomB+1 [no data]
\end{itemize}
\end{frame}

