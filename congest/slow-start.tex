\begin{frame}{``slow start''}
    \begin{itemize}
    \item not very well named
    \vspace{.5cm}
    \item problem is that additive increase doesn't find capacity quickly
    \item exponential(ish) increase to find \textit{initial} window size
    \end{itemize}
\end{frame}

\begin{frame}{``slow start'' on connection begin}
    \begin{itemize}
    \item set window size = 1 packet
    \item increase by one packet for each ACK
    \item \ldots until first packet loss
    \vspace{.5cm}
    \item then revert to additive increase
        \begin{itemize}
        \item actually slower at increasing
        \end{itemize}
    \end{itemize}
\end{frame}

\begin{frame}{``slow start'' later}
    \begin{itemize}
    \item keep track of window size after multiplicative decrease
    \item called \texttt{ssthresh}
        \begin{itemize}
        \item probably variable name in BSD code for this
        \end{itemize}
    \item use slow start when window size lower than ssthresh
    \vspace{.5cm}
    \item but how can that happen?
        \begin{itemize}
        \item need something other than multiplicative decrease
        \end{itemize}
    \end{itemize}
\end{frame}

\begin{frame}{decrease versus reset}
    \begin{itemize}
    \item on duplicate ACK (most common case):
        \begin{itemize}
        \item do multiplicative decrease
        \end{itemize}
    \item on timeout:
        \begin{itemize}
        \item reset window size to 1 packet
        \end{itemize}
    \vspace{.5cm}
    \item after timeout, use ``slow start''
    \item \ldots until ssthresh reached or congestion
        \begin{itemize}
        \item intuition: don't assume halving is enough
        \item intuition: find correct lower window size faster
        \end{itemize}
    \end{itemize}
\end{frame}
