\begin{frame}{diversion: some queuing theory}
    \begin{itemize}
    \item queuing theory: applied probability
    \item talks about how queues work
    \vspace{.5cm}
    \item applies to networks and anything else with ``waiting in line''
    \end{itemize}
\end{frame}

\begin{frame}{queue measurements}
    \begin{itemize}
    \item arrival rate
    \item service time (amount of time after waiting in line)
    \item utilization = arrival rate / service time
    \vspace{.5cm}
    \item if single thing can be processed at a time, then max utilization = 100\%
        \begin{itemize}
        \item higher implies ``infinitely'' long queues
        \end{itemize}
    \end{itemize}
\end{frame}

\begin{frame}{M/M/1/$\infty$ queue}
    \begin{itemize}
    \item next slides: results for M/M/1/$\infty$ queue
    \vspace{.5cm}
    \item M (memoryless) --- random arrival (exponential dist.)
    \item M --- random service time (exponential dist.)
    \item 1 --- one ``server'' (thing that can process packets)
    \item $\infty$ --- unlimited queue length
    \end{itemize}
\end{frame}

\begin{frame}{M/M/1/$\infty$ queue length}
    \begin{itemize}
    \item mean queue length \[\frac{\text{arrival rate}}{\text{service rate} - \text{arrival rate}}\]
    \end{itemize}
\begin{tikzpicture}
\begin{axis}[width=12cm,height=5cm,xlabel=arrival rate (portion of service rate),ylabel=queue length,xmin=0,xmax=1.2,ymin=0,ymax=10]
\addplot[blue,ultra thick,domain=0:1,samples=128]{x/(1-x)};
\end{axis}
\end{tikzpicture}
    \begin{itemize}
    \item<2-> practical implication: \myemph{need to run networks at much less than full utilization}
    \end{itemize}
\end{frame}

\begin{frame}{M/M/1/$\infty$ queue length std. deviation}
    \begin{itemize}
    \item \[\sqrt{\frac{\text{utilization}}{\left(1 - \text{utilization}\right)^2}}\]
    \end{itemize}
\begin{tikzpicture}
\begin{axis}[width=12cm,height=6cm,xlabel=arrival rate (portion of service rate),ylabel=queue length,xmin=0,xmax=1,ymin=0,ymax=10]
\addplot[blue,ultra thick,domain=0:1,samples=128]{sqrt(x/(1-x)^2)};
\end{axis}
\end{tikzpicture}
\end{frame}

\begin{frame}{approx 95th pctile v mean queue length}
\begin{tikzpicture}
\begin{axis}[width=12cm,height=8cm,xlabel=arrival rate (portion of service rate),ylabel=queue length,xmin=0,xmax=1,ymin=0,ymax=10]
\addplot[blue,ultra thick,domain=0:1,samples=128]{x/(1-x)+1.96*sqrt(x/(1-x)^2)};
\addplot[dotted,ultra thick,violet,domain=0:1,samples=128]{x/(1-x)};
\end{axis}
\end{tikzpicture}
\end{frame}
