\begin{frame}{very different congestion control}
    \begin{itemize}
    % FIXME: Fig 32 from TCPCC book
    \item fuller queues $\rightarrow$ higher latency
    \item fuller queues $\rightarrow$ throughput same as window increases
    \vspace{.5cm}
    \item<2-> strategy: monitor throughput/latency to detect full queues
        \begin{itemize}
        \item goal: fill link without making queue grow (much) in size
        \end{itemize}
    \end{itemize}
\end{frame}

\begin{frame}{`Vegas'-style congestion control}
    \begin{itemize}
    \item record ``base'' round-trip time
        \begin{itemize}
        \item connection start or lowest observed
        \end{itemize}
    \item ``ideal'' throughput should be one window / base round trip time
        \begin{itemize}
        \item (Vegas paper calls this ``expected'' throughput)
        \item what would happen with no queuing delay
        \end{itemize}
    \vspace{.5cm}
    \item goal: control what ``ideal'' - actual throughput is
        \begin{itemize}
        \item if 0, queues are probably empty, can increase window
        \item if large, queues are too big, decrease window
        \end{itemize}
    \end{itemize}
\end{frame}

\begin{frame}{BBR-style congestion control}
    \begin{itemize}
    \item \myemph<2>{if queues are empty}, larger window:
    \item latency stays the same and throughput increases
    \vspace{.5cm}
    \item \myemph<3>{if queues are filling}, larger window:
    \item throughput stays the same and latency increases
    \vspace{.5cm}
    \item<4-> observe effect of sending more/fewer packets periodically
    \item<4-> estimate `boundary' based on observed latency/throughput
    \item<4-> keep window size near boundary most of the time
    \end{itemize}
\end{frame}
