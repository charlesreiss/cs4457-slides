\begin{frame}{TCP Tahoe}
    \begin{itemize}
    \item first version of TCP congestion control
    \item track two variables
        \begin{itemize}
        \item \texttt{cwnd} = congestion window
        \item \texttt{ssthresh} = `slow start threshold'
        \end{itemize}
    \item two modes of operation:
    \item \textit{congestion avoidance} (cwnd $\ge$ ssthresh; no known loss)
        \begin{itemize}
        \item represents ``steady state'' --- our focus for now
        \item no loss: cwnd += 1 segment per round-trip-time
        \item on loss: cwnd = $\sim$2 segments; ssthresh = cwnd/2
        \end{itemize}
    \item \textit{recovery} (known loss)
        \begin{itemize}
        \item retransmit packet after third duplicate ACK of previous packet :or timeout
        \end{itemize}
    \item \textit{``slow start''} (cwnd < ssthresh)
        \begin{itemize}
        \item intuition: not in ``steady state'' yet
        \item want to get there  quickly, but with overloading network
        \item we'll give details later
        \end{itemize}
    \end{itemize}
\end{frame}

\begin{frame}{plan}
    \begin{itemize}
    \item really don't want to reset to beginning on loss
        \begin{itemize}
        \item even if we have a way to get to good size quickly
        \end{itemize}
    \item would like to less extreme decrease: cwnd = cwnd/2
    \end{itemize}
\end{frame}
