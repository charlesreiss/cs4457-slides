\begin{frame}{normal backoff}
    \begin{itemize}
    \item problem: what if we have multiple timeouts
    \vspace{.5cm}
    \item let's say timeout is 1 time unit
    \item transmit at 1 time unit, 2 time units, 3 time units, 4 time units, etc.
    \vspace{.5cm}
    \item problem: if the network is overloaded \textit{from retransmissions} won't stop it
        \begin{itemize}
        \item \ldots but window size reduction should make number of packets retransmitted \textit{per connection} low
        \item (so probably not so important with corrected window size management?)
        \end{itemize}
    \end{itemize}
\end{frame}

\begin{frame}{exponential backoff}
    \begin{itemize}
    \item instead of: \\
    \item transmit at 1 time unit, 2 time units, 3 time units, 4 time units, etc.
    \vspace{.5cm}
    \item do something like: \\
    transmit at 1 time unit, 3 time units, 7 time units, 15 time units, etc.
    \end{itemize}
\end{frame}

\begin{frame}{exponential backoff theory}
    \begin{itemize}
    \item for binary exponential backoff 
        \begin{itemize}
        \item timeout for $i$th retransmission is $2^i \times \text{base timeout}$
        \end{itemize}
    \item intuition: avoids overloading network by being a lot less aggressive
        \begin{itemize}
        \item what you `should' do for repeated timeouts due to congestion
        \end{itemize}
    \item not aware of good theoretical results in TCP context
        \begin{itemize}
        \item famous result that this type of backoff is good for things like deciding when to retransmit on shared wired/wireless medium
        \item (Goodman et al, ``On Stability of Ethernet'')
        \end{itemize}
    \end{itemize}
\end{frame}
