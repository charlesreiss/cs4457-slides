\usetikzlibrary{arrows.meta,calc,shapes}
\begin{frame}{showing AIMD}
    \begin{itemize}
    \item slides based on Chiu and Jain, ``Analysis of the Increase and Decrease Algorithms for Congestion Avoidance in Computer Networks'''
    \item 1989 paper
    \vspace{.5cm}
    \item you might notice 1989 is well after TCP was in use
        \begin{itemize}
        \item (kinda deployed without all the theory being developed\ldots
        \item \ldots and it's still not really a solved problem)
        \end{itemize}
    \end{itemize}
\end{frame}

\begin{frame}{picturing sharing}
\begin{tikzpicture}
\tikzset{
    axis/.style={
        draw,ultra thick,-Latex
    },
    normal mark/.style={
        fill=black,
        alt=<5->{fill=black!50},
    },
    normal adjust/.style={
        line width=.5mm,draw=black!50!red,-Latex,
        alt=<5->{draw=black!50!red!50},
    },
}
\begin{scope}[x=1.2cm,y=1.2cm] 
    \path[fill=red!10] (5.5, 0) -- (5.0, 0) -- (0, 5) -- (0, 5.5) -- (5.5, 5.5) -- cycle;
    \path[axis] (0, 0) -- (5.5, 0)
        node[midway,below] {flow 1 bandwidth};
    \path[axis] (0, 0) -- (0, 5.5)
        node[midway,left,align=right] {flow 2\\bandwidth};
    \path[draw,dashed,very thick] (5, 0) -- (0, 5);
    \node[text=red,visible on=<1-3>] at (4, 4) {
        overloaded
    };
    \node[text=black,visible on=<1-3>] at (1.5, 2) {
        underloaded
    };

    \begin{visibleenv}<2>
        \path[draw=black!50,dotted, very thick] (4.5 * 1.2, 2 * 1.2) -- (0, 0);
        \path[line width=1mm,draw=black!50!red,-Latex] (4.5, 2) -- (2.25 * 1.07, 1 * 1.07);
        \path[fill=black] (4.5, 2) circle (2mm);
        \path[fill=black!50] (2.25, 1) circle (2mm);
        \node[anchor=north west,align=left] (incr box) at (4, 4) {
            multiplicative decrease \\
            moves bandwidths along line to origin \\
            example: (90\%, 40\%) to (45\%, 20\%)
        };
        %\draw[violet,very thick] (incr box) -- (4.5, 2+2mm);
    \end{visibleenv}

    \begin{visibleenv}<3>
        \path[draw=black!50,dotted, very thick] (2.25 - 1, 1 - 1) -- (5.5, 1 + 5.5 - 2.25);
        \path[fill=black] (2.25, 1) circle (2mm);
        \path[fill=black!50] (2.25 + 1.3, 1 + 1.3) circle (2mm);
        \node[anchor=north west,align=left] (decr box) at (4, 3) {
            additive decrease \\
            moves bandwidths \\
            at 45-degree angle \\
            example: (45\%, 20\%) to (55\%, 30\%)
        };
    \end{visibleenv}

    \begin{visibleenv}<4->
        \path[normal adjust] (4.5, 2) -- (2.25 * 1.07, 1 * 1.07);
        \path[normal adjust] (2.25 + .14, 1 + .14) -- (2.25 + 1.3 -.14, 1 + 1.3 -.14);
        \path[normal adjust] (2.25 + 1.3 -.14, 1 + 1.3 -.14) -- (1.775*1.07, 1.15*1.07);
        \path[normal adjust] (1.775*1.07, 1.15*1.07) -- (1.775 + 1.3 -.14, 1.15 + 1.3 -.14);
        \path[normal adjust] (1.775 + 1.3 -.14, 1.15 + 1.3 -.14)
            -- (1.5375 * 1.07, 1.225 * 1.07);
        \path[normal adjust] (1.5375 * 1.07, 1.225 * 1.07)
            -- (1.5375 + 1.4 - .14, 1.225 + 1.4 - .14);
        \path[normal adjust] (1.5375 + 1.4 - .14, 1.225 + 1.4 - .14)
            -- (1.4687 * 1.07, 1.325 * 1.07);
        \path[normal mark] (4.5, 2) circle (2mm);
        \path[normal mark] (2.25, 1) circle (2mm);
        \path[normal mark] (2.25 + 1.3, 1 + 1.3) circle (2mm);
        \path[normal mark] (1.775, 1.15) circle (2mm);
        \path[normal mark] (1.775 + 1.3, 1.15 + 1.3) circle (2mm);
        \path[normal mark] (1.5375, 1.225) circle (2mm);
        \path[normal mark] (1.5375 + 1.4, 1.225 + 1.4) circle (2mm);
        \path[normal mark] (1.4687, 1.325) circle (2mm);
    \end{visibleenv}

    \begin{visibleenv}<5->
        \path[draw=black,dashed,very thick] (0,0) -- (5.5, 5.5)
            node[pos=0.75,sloped,above] {equal bandwidth};
    \end{visibleenv}
    \begin{visibleenv}<6>
        \node[anchor=north west,align=left] at (5, 4) {
            multiplicative decrease brings \\
            closer to equal bandwidth line \\
            ~ \\
            additive increase keeps same \\
            distance from line
        };
    \end{visibleenv}
\end{scope}
\end{tikzpicture}
\end{frame}

\begin{frame}{some assumptions we made}
    \begin{itemize}
    \item \ldots that may not always be true
    \vspace{.5cm}
    \item both flows experience drops when network overloaded
    \item same additive increase factor (for 45 degree angle)
    \end{itemize}
\end{frame}
