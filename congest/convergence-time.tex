\begin{frame}{exercise: convergence time (1)}
    \begin{itemize}
    \item suppose: 50 ms round trip time
    \item initially sending at 600 packets/second
        \begin{itemize}
        \item $\approx 0.9$Mbyte/sec with 1500 byte packets
        \end{itemize}
    \item optimal rate is 10000 packets/second
        \begin{itemize}
        \item $\approx 15$Mbyte/sec with 1500 byte packets
        \end{itemize}
    \item `standard' TCP increase of 1 packet/RTT
    \item how long to get there?
    \vspace{.5cm}
    \item<2-> current: 30 packets/RTT (= window size 30)
    \item<2-> need to get to: 500 packets/RTT
    \item<2-> will take $500-30=470$ round trips $\approx$ 23500 ms $\approx$ 24 s
    \end{itemize}
\end{frame}

\begin{frame}{fixing bad convergence time}
    \begin{itemize}
    \item TCP's additive increase is very slow for ``high bandwidth-delay'' networks
    \item two things make this better today:
    \vspace{.5cm}
    \item more adaptive increase for modern TCP variants
        \begin{itemize}
        \item e.g. FAST TCP, CUBIC TCP, \ldots
        \item heuristics to increase faster when appropriate
        \end{itemize}
    \item not in additive increase mode at start of connection
        \begin{itemize}
        \item ``slow start'' we'll talk about later
        \end{itemize}
    \end{itemize}
\end{frame}
