\begin{frame}{self-clocking and dup-ACKs}
    \begin{itemize}
    \item without losses, sender sends one new packet per ACK
    \item keeps number of packets in network constant
    \item but duplicate ACKs are exception (say window size 6):
    \end{itemize}
{\fontsize{10}{11}\selectfont
\begin{tabular}{llll}
recv'd & sent & count of packets in flight \\ \hline
--- & data 0-5 & 6 \\
ACK 0 & ~ & 5 \\
~ & data 6 & 6 \\
ACK 1 & ~ & 5 \\
~ & data 7 & 6 \\
ACK 1 & ~  & 5 \\
ACK 1 & ~  & 4 \\
ACK 1 & ~ & 3 \\
~ & data 2 & 4 \\
ACK 1 & ~ & 3 \\
\end{tabular}
}
\end{frame}

\begin{frame}{alternate explanation}
    \begin{itemize}
    \item sender stopped sending while receiving duplicate ACKs
    \item but we know \textit{most messages got there}
    \item means our usage of network doesn't reflect out window size
    \end{itemize}
\end{frame}

\begin{frame}{TCP's fast retransmission}
\begin{itemize}
\item on third duplicate ACK:
\vspace{.5cm}
\item resend packet,
\item do multiplicative decrease, AND THEN
\item temporarily add 2 packet to window for each dup ACK
    \begin{itemize}
    \item send packets to replace received packet
    \item (if allowed by multiplicative-decreased window)
    \end{itemize}
\item reset window size back when later ACK
\end{itemize}
\end{frame}

\begin{frame}{self-clocking and fast retransmit}
\begin{itemize}
\item adjust window size to keep packets in flight constant:
\end{itemize}
{\fontsize{9}{10}\selectfont
\begin{tabular}{llll}
recv'd & sent & count packets in flight & send window size (range) \\ \hline
--- & data 0-5 & 6 & 6 (0-5)\\
ACK 0 & ~ & 5 & 6 (1-6) \\
~ & data 6 & 6 & 6 (1-6)\\
ACK 1 & ~ & 5 & 6 (2-7)\\
~ & data 7 & 6 & 6 (2-7)\\
ACK 1 & ~  & 5 & 6 (2-7)\\
ACK 1 & ~  & 4 & 6 (2-7)\\
ACK 1 & ~ & 3 & 6 (2-7) \\
~ & data 2 & 4 & \myemph{8} (2-9)\\
~ & \myemph{data 8} & 5 & 8 (2-9)\\
~ & \myemph{data 9} & 6 & 8 (2-9)\\
ACK 1 & ~ & 5 & \myemph{9} (2-10)  \\
~ 1 & \myemph{data 10} & 5 & \myemph{9} (2-10)  \\
\end{tabular}
}
\end{frame}
