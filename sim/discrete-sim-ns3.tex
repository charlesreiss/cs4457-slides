\begin{frame}\frametitle{NS-3}
    \begin{itemize}
    \item discrete event simulator planned for AIMD assignment
    \item written in C++
        \begin{itemize}
        \item (yes, I know it's not the most familiar language)
        \item (obvious alternative simulators aren't in better languages\ldots)
        \end{itemize}
    \item create simulations by writing C++ programs
    \end{itemize}
\end{frame}

\begin{FragileFrame}
\frametitle{an NS-3 event handler}
\providecommand{\myemphA}[1]{\myemph<2>{#1}}
\providecommand{\myemphB}[1]{\myemph<3>{#1}}
\providecommand{\myemphC}[1]{\myemph<4>{#1}}
\begin{Verbatim}[fontsize=\fontsize{9}{10},commandchars=\\QX]
bool \myemphAQPointToPointChannel::TransmitStartX(
    Ptr<const Packet> p,
    Ptr<PointToPointNetDevice> src,
    Time txTime
) {
  // ...
  uint32_t wire = src == \myemphBQm_Xlink[0].\myemphBQm_Xsrc ? 0 : 1;

  \myemphCQ\myemphAQSimulator::ScheduleWithContextXX(
    \myemphBQm_Xlink[wire].m_dst->GetNode()->GetId(),
    txTime + \myemphBQm_Xdelay,
    &\myemphAQPointToPointNetDevice::ReceiveX, \myemphBQm_Xlink[wire].m_dst, p->Copy());

  // Call the tx anim callback on the net device
  \myemphBQm_XtxrxPointToPoint(p, src, ...)
  return true;
}
\end{Verbatim}
\begin{tikzpicture}[overlay,remember picture]
    \coordinate (place) at ([yshift=-1cm,xshift=-1cm]current page.north east);
    \tikzset{%
        box/.style={at=(place),anchor=north east,draw=orange!70!black,very thick,align=left,fill=white,font=\small}
    }
    \begin{visibleenv}<2>
        \node[box] {X::Y = Y method/variable of X class};
    \end{visibleenv}
    \begin{visibleenv}<3>
        \node[box] {%
            convention: member variables with \texttt{m\textunderscore} \\
            (C++ member variable $\sim$ Java instance variable)
        };
    \end{visibleenv}
    \begin{visibleenv}<4>
        \node[box] {%
            setup future event; args: \\
            context (for logging mostly) \\
            time event will trigger \\
            method to run + arguments to pass
        };
    \end{visibleenv}
\end{tikzpicture}
\end{FragileFrame}

\begin{FragileFrame}
\frametitle{sample NS-3 simulation --- setup (1)}
{\scriptsize \texttt{ns3/examples/tcp/tcp-bulk-send.cc}}
\begin{Verbatim}[fontsize=\fontsize{9}{10}]
// nodes = routers or endpoints
NodeContainer nodes;
nodes.Create(2);

// create simulated point-to-point link
    // also supported: multi-access links
PointToPointHelper pointToPoint;
pointToPoint.SetDeviceAttribute("DataRate", StringValue("500Kbps"));
pointToPoint.SetChannelAttribute("Delay", StringValue("5ms"));

// setup emulated NICs (which have queues, etc.)
NetDeviceContainer devices;
devices = pointToPoint.Install(nodes);

// setup emulated TCP/IP implementation
InternetStackHelper internet;
internet.Install(nodes);

...
\end{Verbatim}
\end{FragileFrame}

\begin{FragileFrame}
\frametitle{sample  NS-3 simulation --- setup (2)}
{\scriptsize \texttt{ns3/examples/tcp/tcp-bulk-send.cc}}
\begin{Verbatim}[fontsize=\fontsize{9}{10}]
// Simulated "applications" that send/receive data
BulkSendHelper source("ns3::TcpSocketFactory", InetSocketAddress(i.GetAddress(1), port));
source.SetAttribute("MaxBytes", UintegerValue(10000000));
ApplicationContainer sourceApps = source.Install(nodes.Get(0));
sourceApps.Start(Seconds(0));
sourceApps.Stop(Seconds(10));
PacketSinkHelper sink("ns3::TcpSocketFactory", InetSocketAddress(Ipv4Address::GetAny(), port));
ApplicationContainer sinkApps = sink.Install(nodes.Get(1));
sinkApps.Start(Seconds(0));
sinkApps.Stop(Seconds(10));
\end{Verbatim}
\end{FragileFrame}

\begin{FragileFrame}
\frametitle{sample NS-3 simulation --- setup (3)}
{\scriptsize \texttt{ns3/examples/tcp/tcp-bulk-send.cc}}
\begin{Verbatim}[fontsize=\fontsize{9}{10}]
AsciiTraceHelper ascii;

// produces text trace file of simulator events
pointToPoint.EnableAsciiAll(ascii.CreateFileStream("tcp-bulk-send.tr"));

// produces PCAP files you can open in Wireshark
pointToPoint.EnablePcapAll("tcp-bulk-send", false);
\end{Verbatim}
\end{FragileFrame}

% FIXME: show screenshot of wireshark on output
% FIXME: show .tr file

\begin{FragileFrame}
\frametitle{running example: --help}
\begin{Verbatim}[fontsize=\fontsize{9}{10},commandchars=\\\{\}]
$ \myemph{./ns3 run examples/tcp/tcp-bulk-send -- --help}
[0/2] Re-checking globbed directories...
ninja: no work to do.
tcp-bulk-send [Program Options] [General Arguments]

Program Options:
    --tracing:   Flag to enable/disable tracing [false]
    --maxBytes:  Total number of bytes for application to send [0]

General Arguments:
    --PrintGlobals:              Print the list of globals.
    --PrintGroups:               Print the list of groups.
    --PrintGroup=[group]:        Print all TypeIds of group.
    --PrintTypeIds:              Print all TypeIds.
    --PrintAttributes=[typeid]:  Print all attributes of typeid.
    --PrintVersion:              Print the ns-3 version.
    --PrintHelp:                 Print this help message.
\end{Verbatim}
\end{FragileFrame}

\begin{FragileFrame}
\frametitle{running example}
\begin{Verbatim}[fontsize=\fontsize{9}{10},commandchars=\\\{\}]
$ \myemph{./ns3 run examples/tcp/tcp-bulk-send -- --maxBytes=1000000 --tracing=true}
[0/2] Re-checking globbed directories...
ninja: no work to do.
Total Bytes Received: 566016

$ \myemph{wireshark tcp-bulk-send-0-0.pcap}
$ \myemph{wireshark tcp-bulk-send-1-0.pcap}
$ \myemph{wireshark tcp-bulk-send-2-0.pcap}
$ \myemph{wireshark tcp-bulk-send-2-1.pcap}
\end{Verbatim}
    \hrule
\begin{itemize}
\item packet traces for each simulated interface: (host)-(interface).pcap
\item also .tr file with all events
\item only 566016 bytes received because 10 second simulation only
    \begin{itemize}
    \item 500Kbit/s = 62KByte/s @ 10 s = about 620KByte
    \end{itemize}
\end{itemize}
\end{FragileFrame}
