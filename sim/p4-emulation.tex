\begin{frame}
\frametitle{emulation}
\begin{itemize}
\item P4 assignment virtual machine
\item based on \textit{mininet}
\item creates ``virtual'' network devices on Linux system
    \begin{itemize}
    \item no actual hardware for network device
    \item connected to program (implementing P4 switch) and/or other virtual network devices
    \end{itemize}
\item emulation runs `as fast as it can':
    \begin{itemize}
    \item performance based on speed of programs talking to simulated devices
    \end{itemize}
\item could add artificial delays/bandwidht constraints to emulate more realistic networks
    \begin{itemize}
    \item make things more realistic by making things slower?
    \end{itemize}
\end{itemize}
\end{frame}

\begin{frame}
\frametitle{emulation constraints}
\begin{itemize}
\item does not reflect real network deployment in important ways
    \begin{itemize}
        \item especially for evaluating congestion control\ldots
    \end{itemize}
\vspace{.5cm}
\item performance of one simulated hosts affects others because shared machine
\item performance will be unrealistically low for large networks
\item round-trip timing affected a lot by packet processing speeds
\item speed of packets being sent limited by how fast CPU is
    \begin{itemize}
    \item never going to get to `real' speed of hardware-accelerated switches
    \end{itemize}
\end{itemize}
\end{frame}
