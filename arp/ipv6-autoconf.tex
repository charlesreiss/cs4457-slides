\begin{frame}{IPv6 autoconfiguration}
    \begin{itemize}
    \item in IPv4, autoconfiguration ``bolted on''
        \begin{itemize}
        \item one protocol to be assigned address (DHCP)
        \item one protoocl to communicate IP address to other nodes on network (ARP)
        \end{itemize}
    \item IPv6 was designed later, so they thought about it early
    \end{itemize}
\end{frame}

\begin{frame}{big network address assignment}
    \begin{itemize}
    \item IPv6 local networks are typically /64s
        \begin{itemize}
        \item $2^{64}$ address available for local network
        \end{itemize}
    \item why so big? allow easy address assignment
    \item StateLess Address Auto Configuration (SLAAC)
    \end{itemize}
\end{frame}

\begin{frame}{MAC-address based address assignment}
    \begin{itemize}
    \item let's say my local network is 2001:db8:4999:3333::/64
    \end{itemize}
{\fontsize{12}{13}\selectfont
\begin{tabular}{ll}
MAC address & IPv6 address \\
\tt \myemph<2>{11:22:33:44:55:66} & \tt 2001:db8:4999:3333:\myemph<2>{1122:33}ff:fe\myemph<2>{44:5566} \\
\tt \myemph<2>{01:A0:B3:CC:DD:FF} & \tt 2001:db8:4999:3333:\myemph<2>{01a0:b3}ff:fe\myemph<2>{cc:ddff} \\
\ldots & \ldots \\
\end{tabular}}
\end{frame}

\begin{frame}[fragile]{}
\begin{Verbatim}[fontsize=\fontsize{9}{10}\selectfont]
Network Working Group                                          T. Narten
Request for Comments: 3041                                           IBM
Category: Standards Track                                      R. Draves
                                                      Microsoft Research
                                                                                                                  January 2001
                                                                                                                     Privacy Extensions for Stateless Address Autoconfiguration in IPv6

...
Abstract

                                  ........  Use of the extension causes
   nodes to generate global-scope addresses from interface identifiers
   that change over time, even in cases where the interface contains an
   embedded IEEE identifier.  Changing the interface identifier (and the
   global-scope addresses generated from it) over time makes it more
   difficult for eavesdroppers and other information collectors to
   identify when different addresses used in different transactions
   actually correspond to the same node.
\end{Verbatim}
\end{frame}

\begin{frame}{late timeline}
\begin{itemize}
\item privacy extensions weren't default until
    \begin{itemize}
    \item MacOS X 10.7 (2011)
    \item Windows Vista (2007)
    \item Ubuntu 12.04 (2012)
    \end{itemize}
\end{itemize}
\end{frame}

% FIXME: wireshark example
\begin{frame}{SLAAC}
\begin{itemize}
\item uses as ICMPv6 (same as for neighbor discovery)
\item two modes:
    \begin{itemize}
    \item when there's a router (get \textit{global} addresses)
    \item when there's only a local network (get \textit{link-local} addresses)
    \end{itemize}
\end{itemize}
\end{frame}

\begin{frame}{router adverisements}
\begin{tiemize}
\item nodes send a ICMPv6 \textit{Router Solicitation} message
\item receive back a ICMPv6 \textit{Router Advertisement} which can have:
\vspace{.5cm}
\item \myemph<2>{prefix information}
    \begin{itemize}
    \item nodes choose address starting with prefix
    \item check for duplicates using neighbor discovery
    \end{itemize}
\item \myemph<3>{DNS information}
\item \myemph<4>{``managed configuration'' flag}
    \begin{itemize}
    \item nodes use DHCPv6 to get configuration
    \end{itemize}
\item \myemph<5>{``other configuration'' flag}
    \begin{itemize}
    \item nodes choose address using prefix, get additional information from DHCPv6
    \end{itemize}
\end{itemize}
\end{frame}
