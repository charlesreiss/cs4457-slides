\usetikzlibrary{arrows.meta,calc,fit,matrix,shapes}
\begin{frame}{ARP/ND protocols}
    \begin{itemize}
    \item filling in tables dynamically?
    \item key idea: ask \textit{everyone} on network
    \item entity with that IP address responds
    \vspace{.5cm}
    \item IPv4: Address Resolution Protocol (ARP)
    \item IPv6: ICMPv6 Neighbor Discovery (ND)
        \begin{itemize}
        \item ICMP = Internet Control Message Protocol
        \end{itemize}
    \end{itemize}
\end{frame}

\begin{frame}{ARP messages}
\begin{itemize}
\item suppose router IP address 10.0.2.15 and MAC address 02:\ldots:DE needs to find out
    that 10.0.2.2 uses 03:\ldots:EE
\vspace{.5cm}
\item 02:\ldots:DE$\rightarrow$FF:FF:FF:FF:FF:FF: request 10.0.2.2 \\
      tell 10.0.2.15 (=02:\ldots:DE)
    \begin{itemize}
    \item FF:FF:FF:FF:FF:FF = broadcast (send to all on network)
    \end{itemize}
\item 03:\ldots:EE$\rightarrow$02:\ldots:DE: reply 10.0.2.2=03:\ldots:EE
    \\ tell 10.0.2.15=02:\ldots:DE
\end{itemize}
\end{frame}

\begin{frame}[fragile]{ARP message format}
\begin{tikzpicture}
    \tikzset{
        box/.style={draw,thick},
        box unused/.style={draw,thick,pattern=north west lines},
        box label/.style={midway,font=\small,align=center},
        box label flags/.style={midway,font=\fontsize{8}{9}\selectfont,align=center},
        hi on/.style={alt=<#1>{ultra thick,fill=red!10}},
        explain box 1/.style={draw=red,line width=0.8mm,fill=white,anchor=center,at={(explain box loc 1)},align=center},
        explain box 2/.style={draw=red,line width=0.8mm,fill=white,anchor=center,at={(explain box loc 2)},align=center},
        explain box 3/.style={draw=red,line width=0.8mm,fill=white,anchor=center,at={(explain box loc 3)},align=center},
    }
    \begin{scope}[x=4.7mm,y=7mm]
        \coordinate (explain box loc 1) at (16, -3.1);
        \coordinate (explain box loc 2) at (16, -5.1);
        \coordinate (explain box loc 3) at (16, -1.1);
        \draw[very thick,Latex-Latex] (0, 1.1) -- ++(32, 0) node[font=\small,midway,above]
            {32 bits};
        \path[shading=axis,top color=white,bottom color=black!20] (0, 1) rectangle (32, 0)
            node[box label] {(lower layer header)};
        \path[box,hi on=0] (0, 0) rectangle (16, -1) node[box label] {HW address type};
        \path[box,hi on=2] (16, 0) rectangle (32, -1) node[box label] {`protocol' address type};
        \path[box,hi on=0] (0, -1) rectangle ++(8, -1) node[box label] {HW address length};
       \path[box,hi on=2] (8, -1) rectangle ++(8, -1) node[box label] {`protocol' address len};
        \path[box,hi on=0] (16, -1) rectangle ++(16, -1) node[box label] {opcode (request or reply)};
        \path[box,hi on=0] (0, -2) rectangle ++(32, -1.5) node[box label] {sender HW addr};
        \path[box,hi on=0] (0, -3.5) rectangle ++(32, -1.5) node[box label] {sender protocol addr};
        \path[box,hi on=3] (0, -5) rectangle ++(32, -1.5) node[box label] {dest HW addr};
        \path[box,hi on=0] (0, -6.5) rectangle ++(32, -1.5) node[box label] {dest protocol addr};
        \foreach \offset in {-2.75,-4.25,-5.75, -7.25} {
            \draw[line width=1mm,white] (-.2, \offset - .05) -- ++(.4, .1);
            \draw[line width=1mm,white] (32 -.2, \offset - .05) -- ++(.4, .1);
            \draw[very thick,black] (-.2, \offset) -- ++(.4, .1);
            \draw[very thick,black] (32 -.2, \offset) -- ++(.4, .1);
            \draw[very thick,black] (-.2, \offset-.1) -- ++(.4, .1);
            \draw[very thick,black] (32 -.2, \offset-.1) -- ++(.4, .1);
        }
    \end{scope}
    \begin{visibleenv}<2>
    \node[explain box 2] {
        protocol typically = IPv4 \\
        so address len = 4 \\
        ~ \\
        seems like you could change this for IPv6 \\
        but instead IPv6 uses its own protocol
    };
    \end{visibleenv}
    \begin{visibleenv}<3>
    \node[explain box 3] {
        to keep format consistent \\
        between request/replies \\
        \myemph{always have destination HW address} \\
        kept at FF:\ldots:FF if unknown
    };
    \end{visibleenv}
\end{tikzpicture}
\end{frame}

\begin{frame}{ARP messages (revisited)}
\begin{itemize}
\item suppose router IP address 10.0.2.15 and MAC address 02:\ldots:DE needs to find out
    that 10.0.2.2 uses 03:\ldots:EE
\vspace{.5cm}
\item 02:\ldots:DE$\rightarrow$FF:FF:FF:FF:FF:FF: request 10.0.2.2 \\
      tell 10.0.2.15 (=02:\ldots:DE)
      \begin{itemize}
      \item \myemph{everyone who receives this can add 10.0.2.15=02:\ldots:DE to neighbor table}
      \end{itemize}
\item 03:\ldots:EE$\rightarrow$02:\ldots:DE: reply 10.0.2.2=03:\ldots:EE
    \\ tell 10.0.2.15=02:\ldots:DE
      \begin{itemize}
      \item \myemph{everyone who receives this can add 10.0.2.2=03:\ldots:EE to neighbor table}
      \end{itemize}
\end{itemize}
\end{frame}

\begin{frame}{ICMPv6 ND}
    \begin{itemize}
    \item IPv6 uses different protocol for this
    \item \ldots but mostly works the same
    \vspace{.5cm}
    \item differences:
    \item sent as IPv6 packets
    \item requests sent to special multicast address
        \begin{itemize}
        \item goal: allow nodes to easily ignore irrelevant requests
        \end{itemize}
    \item different names:
        \begin{itemize}
        \item request = solicitiation
        \item reply = advertisement
        \end{itemize}
    \end{itemize}
\end{frame}
