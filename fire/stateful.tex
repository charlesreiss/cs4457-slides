\begin{frame}{motivation for stateful firewall}
    \begin{itemize}
    \item common policy: allow outgoing connections only
    \vspace{.5cm}
    \item prior approach:
        \begin{itemize}
        \item drop incoming non-TCP, or TCP SYN
        \end{itemize}
    \item problems:
        \begin{itemize}
        \item disallowing UDP-based protocols (example: DNS over UDP, HTTP/3)
        \item disallowing normal ICMP (example: ICMP ping replies)
        \item allowing unsolicited TCP packets
        \end{itemize}
    \end{itemize}
\end{frame}

\begin{frame}{keeping state}
    \begin{itemize}
    \item outgoing connections only?
    \item really want to track list of \textit{connections}
    \vspace{.5cm}
    \item most common form of \textit{stateful firewall}
    \end{itemize}
\end{frame}

\begin{frame}{connection tracking?}
    \begin{itemize}
    \item simple idea for TCP:
    \item see SYN: add (TCP, source IP+port, dest IP+port) to table
    \item see FIN, FIN+ACK, ACK: remove row from table after timeout
    \item (plus other timeouts, error cases, etc.)
    \vspace{.5cm}
    \item mirrors TCP state machine
    \end{itemize}
\end{frame}

\begin{frame}{Linux conntrack}
    \begin{itemize}
    \item in-kernel table of active ``connections''
    \item includes notion of connections for UDP, ICMP, etc.
        \begin{itemize}
        \item heuristic guesses since protocol has no connect/close operation
        \end{itemize}
    \item maintains table of (proto, source host+port, dest host+port)
    \item packets marked with connection state
        \begin{itemize}
        \item new, established, \myemph<2>{related}, invalid
        \end{itemize}
    \end{itemize}
\end{frame}
