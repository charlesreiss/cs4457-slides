\usetikzlibrary{arrows.meta,calc,shapes.misc,shapes}
\begin{frame}{NAT idea}
\begin{tikzpicture}
\tikzset{network/.style={draw,very thick,cloud,aspect=2}}
\node[align=center,network] (outside) at (0, 0){external};
\node[draw,very thick,minimum height=2cm,align=center](router) at (5, 0) {
    router \\
    {\fontsize{9}{10}\tt 203.0.113.43} \\
    {\fontsize{9}{10}\tt 192.168.1.1} 
};
\node[align=center,network] (inside) at (9.5, 0){internal\\\fontsize{9}{10}\tt 192.168.1.*};
\foreach \x/\v in {2/2, 1/3, 0/4, -1/5, -2/6} {
    \node[draw,very thick,circle,
        label={[overlay,font=\fontsize{8}{9}\selectfont]right:192.168.1.\v}
    ] (i-\x) at (12, \x) {};
    \draw (inside) -- (i-\x);
}
\begin{pgfonlayer}{bg}
\begin{scope}[line width=1mm,black!50]
    \draw (outside) -- (router);
    \draw (router) -- (inside);
\end{scope}
\end{pgfonlayer}
\tikzset{
    box/.style={draw,ultra thick,align=center,fill=white,font=\fontsize{13}{14}\selectfont}
}
\node[box] (before) at ([yshift=-3cm]$(outside.east)!0.3!(router)$) {
    1.2.3.4:443 \\ $\rightarrow$ \\ \myemph{203.0.113.43}:54923
};
\node[box] (after) at ([yshift=-3cm]$(router)!0.8!(inside.west)$) {
    1.2.3.4:443 \\ $\rightarrow$ \\ \myemph{192.168.1.4}:39129
};
\draw[dotted,thick] (before.north) -- (before.north |- outside.east);
\draw[dotted,thick] (after.north) -- (after.north |- outside.east);

\end{tikzpicture}
\end{frame}

\begin{frame}{network address translation (NAT) (1)}
    \begin{itemize}
    \item internal network uses private IPv4 addresses
    \item need to translate to (fewer) public IPv4 addresses
    \item add `internal IP+port' to connection tracking state
    \item use that info to rewrite packets
    \end{itemize}
\fontsize{10}{11}\selectfont\tt
\begin{tabular}{l|l|l|l|l}
proto & remote IP + port & public IP  + port & internal IP + port \\
TCP & 128.143.67.8:443 & 198.51.100.17:43232 & 192.168.1.54:59549 \\ 
TCP & 128.143.67.8:443 & 198.51.100.17:59948 & 192.168.1.13:59549 \\ 
UDP & 216.239.32.10:53 & 198.51.100.17:39554 & 192.168.1.2:31923 \\ 
\end{tabular}
\end{frame}


\begin{frame}{NAT illusion}
    \begin{itemize}
    \item NAT illusion:
    \item private IP address communicating directly with public IP
    \vspace{.5cm}
    \item inside network, talking to outside:
        \begin{itemize}
        \item use private local address
        \item use public remote address
        \item never see router's address
        \end{itemize}
    \item outside network, talking to inside
        \begin{itemize}
        \item use public local address
        \item use router's public address
        \end{itemize}
    \end{itemize}
\end{frame}

\begin{frame}{network address translation (NAT) (2)}
    \begin{itemize}
    \item distribute private IPv4 addresses from on internal network
        \begin{itemize}
        \item most common use case: home routers for IPv4
        \item also used by many companies, ISPs, etc.
        \end{itemize}
    \item can support tons of connections with one IPv4 address
        \begin{itemize}
        \item recall: different (remote IP+port, local port) = diff connnection
        \end{itemize}
    \item can use multiple public IPs if risk of running out of port numbers
        \begin{itemize}
        \item likely common in big NAT installations
        \end{itemize}
    \end{itemize}
\end{frame}

\begin{frame}{endpoint-independent mapping}
    \begin{itemize}
    \item recall: UDP supports receiving from multiple places with on socket
    \item so maybe our table entry should be:
    \end{itemize}
\begin{tabular}{l|l|l|l|l}
proto & remote IP + port & public IP  + port & internal IP + port \\
UDP & \myemph{(any)} & 198.51.100.17:39554 & 192.168.1.2:31923 \\ 
\end{tabular}
    \begin{itemize}
    \item also might make sense for TCP
        \begin{itemize}
        \item some applications may deliberate reuse source port
        \end{itemize}
    \item most common NAT implementation, but limits number of hosts that can be supported
    \vspace{.5cm}
    \item (can also achieve this effect by choose public IP+port consistently)
        \begin{itemize}
        \item example: hash of private IP + port
        \end{itemize}
    \end{itemize}
\end{frame}

\begin{frame}{mapping versus filtering}
    \begin{itemize}
    \item packet filtering can be separate from translation
    \end{itemize}
NAT table: \\
\small
\begin{tabular}{l|l|l|l|l}
proto & remote IP + port & public IP  + port & internal IP + port \\
UDP & \myemph{(any)} & 198.51.100.17:39554 & 192.168.1.2:31923 \\ 
\end{tabular} \\
connection table for filtering:\\
\small
\begin{tabular}{l|l|l|l|l}
proto & remote IP + port & public IP  + port & internal IP + port \\
UDP & 203.0.113.34:4444 & 198.51.100.17:39554 & 192.168.1.2:31923 \\ 
UDP & 192.0.2.99:8999 & 198.51.100.17:39554 & 192.168.1.2:31923 \\ 
\end{tabular}
\end{frame}


\begin{frame}{running servers inside NAT (1)}
    \begin{itemize}
    \item so far: can't accept connections on the private network
    \vspace{.5cm}
    \item simple solution: sysadmin configures port forwarding
    \item basically static connection table entries:
    \end{itemize}
\small
\begin{tabular}{l|l|l|l|l}
proto & remote IP + port & public IP  + port & internal IP + port \\
TCP & (any) & 198.51.100.17:443 & 192.168.1.100:443 \\ 
TCP & (any) & 198.51.100.17:22 & 192.168.1.100:22 \\ 
\end{tabular}
\end{frame}

\begin{frame}<1>[label=servInNat]{running servers inside NAT (2)}
    \begin{itemize}
    \item often want to accept connections not configured by sysadmin
    \item example: direct video call between two users
        \begin{itemize}
        \item would be better to send directly
        \end{itemize}
    \vspace{.5cm}
    \item some classes of solution:
        \begin{itemize}
        \item \myemph<2>{ask router to add table entry}
        \item \myemph<3>{coordinate with other end to setup connection}
        \item \myemph<4>{go through relay}
        \end{itemize}
    \item extra issues:
        \begin{itemize}
        \item \myemph<5>{two hosts behind same NAT? nested NATs?}
        \end{itemize}
    \end{itemize}
\end{frame}

\againframe<2>{servInNat}

\begin{frame}{router protocols}
    \begin{itemize}
    \item several (related) protocols for router-helped NAT traversal
        \begin{itemize}
        \item Port Control Protocol, NAT Port Mapping Protocol, UPnP Internet Gateway Protocol
        \end{itemize}
    \item discovered via UDP multicast and/or DHCP
    \item all provide:
        \begin{itemize}
        \item way of learning next external IP
        \item way to requesting externally accessible port
        \end{itemize}
    \item typical have lease times for external ports
        \begin{itemize}
        \item host is expected to renew periodically
        \end{itemize}
    \end{itemize}
\end{frame}

\againframe<3>{servInNat}

\begin{frame}{using learned mappings}
    \begin{itemize}
    \item \myemph{if endpoint-independent NAT}:
    \item choose local (private) IP address + port
    \item contact external server to learn corresponding public IP address + port
        \begin{itemize}
        \item protocol for this: STUN (RFC 5389)
        \end{itemize}
    \item communicate that IP address + port to other end
        \begin{itemize}
        \item example: via video call setup server
        \end{itemize}
    \item both connect to other IP address + port
    \end{itemize}
\end{frame}

\begin{frame}{complications (1)}
    \begin{itemize}
    \item potential issue: firewall rule might block unsolicited packets
        \begin{itemize}
        \item might need entry in connection table to get packet not dropped
        \end{itemize}
    \item for UDP: both ends send packet first to setup firewall rule
    \item for TCP: simulatenous connection attempts may work
        \begin{itemize}
        \item TCP state machine supports `simulatenous open'
        \item probably one SYN needs to be resent
        \end{itemize}
    \end{itemize}
\end{frame}

\begin{frame}{complications (2)}
    \begin{itemize}
    \item potential issue: ``clever'' NATs corrupt addresses
    \vspace{.5cm}
    \item one bad NAT idea: automatically change private IP + port to public IP + port in packet contents
    \item \ldots just look for matching bytes and substitute them! (don't worry about understanding the protocol)
    \item shouldn't do this: probably corrupt downloads/break things
        \begin{itemize}
        \item some NATs did/do it anyways
        \end{itemize}
    \vspace{.5cm}
    \item STUN workaround: `obfuscate' address+port with XOR
    \end{itemize}
\end{frame}

\againframe<4>{servInNat}

\begin{frame}{external relays}
    \begin{itemize}
    \item RFC 8656: Traversal Using Relays around NAT (TURN)
    \item support for setting up relays dynamically
    \vspace{.5cm}
    \item STUN + TURN often used with WebRTC
    \item WebRTC = web browser video/audio-conference support
    \item video conferencing provider might run STUN/TURN servers
        \begin{itemize}
        \item info passed to WebRTC-based webapp
        \end{itemize}
    \end{itemize}
\end{frame}

\begin{frame}{exercise: nested NATs}
    \begin{itemize}
    \item how do these solutions deal with nested network address translation?
        \begin{itemize}
        \item when, if ever, will these solutions break
        \end{itemize}
    \vspace{.5cm}
    \item 1. A and B ask NAT for external port
    \item 2. A and B learn external IP+port and use simulatenous UDP connection
    \item 3. A and B use external relay
    \end{itemize}
\end{frame}

\begin{frame}{aside: address reuse in NAT}
    \begin{itemize}
    \item with nested network address translation might have\ldots
    \item home network (10/8) $\leftrightarrow$ ISP's NAT (10/8) $\leftrightarrow$ internet (`real')
    \vspace{.5cm}
    \item conceptually nothing prevents this from working
        \begin{itemize}
        \item NAT box translates 10.x.y.z addresses to different 10.a.b.c address
        \item routing with 10.w.x.y `gateway' also specifies interface
        \item (yes, can't contact other IPs in ISP's 10/8 network easily from home network, but probably don't care)
        \end{itemize}
    \item \ldots but confuses a lot of implementations
    \item reason for special CGNAT IP space 100.64/10
        \begin{itemize}
        \item supposed to only be used by devices that can handling nesting like this
        \end{itemize}
    \end{itemize}
\end{frame}
