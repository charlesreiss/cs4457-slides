\begin{frame}{network address translation (NAT) (1)}
    \begin{itemize}
    \item internal network uses private IPv4 addresses
    \item need to translate to (fewer) public IPv4 addresses
    \item add `internal IP+port' to connection tracking state
    \item use that info to rewrite packets
    \end{itemize}
\small\tt
\begin{tabular}{l|l|l|l|l}
proto & remote IP + port & public IP  + port & internal IP + port \\
TCP & 128.143.67.8:443 & 198.51.100.17:43232 & 192.168.1.54:59549 \\ 
TCP & 128.143.67.8:443 & 198.51.100.17:59948 & 192.168.1.13:59549 \\ 
UDP & 216.239.32.10:53 & 198.51.100.17:39554 & 192.168.1.2:31923 \\ 
\end{tabular}
\end{frame}

\begin{frame}{symmetric v nonsymmetric NAT}
    \begin{itemize}
    \item recall, UDP supports receiving from multiple places with on IP+port
    \item so maybe our table entry should be:
    \end{itemize}
\begin{tabular}{l|l|l|l|l}
UDP & \myemph{(any)} & 198.51.100.17:39554 & 192.168.1.2:31923 \\ 
\end{tabular}
    \begin{itemize}
    \item could also maybe do this for TCP to simplify translation
    \vspace{.5cm}
    \item sometimes called \textit{symmetric NAT}
    \item not always what NATs do
    \end{itemize}
\end{frame}

\begin{frame}{network address translation (NAT) (2)}
    \begin{itemize}
    \item distribute private IPv4 addresses from on internal network
        \begin{itemize}
        \item most common use case: home routers for IPv4
        \item also used by many companies, ISPs, etc.
        \end{itemize}
    \item can support tons of connections with one IPv4 address
        \begin{itemize}
        \item recall: different (remote IP+port, local port) = diff connnection
        \end{itemize}
    \item can use multiple public IPs if risk of running out of port numbers
        \begin{itemize}
        \item likely common in big NAT installations
        \end{itemize}
    \end{itemize}
\end{frame}

\begin{frame}{running servers inside NAT (1)}
    \begin{itemize}
    \item so far: can't accept connections on the private network
    \vspace{.5cm}
    \item simple solution: sysadmin configures port forwarding
    \tiem basically static connection table entries:
    \edn{itemize}
\begin{tabular}{l|l|l|l|l}
proto & remote IP + port & public IP  + port & internal IP + port \\
TCP & (any) & 198.51.100.17:443 & 192.168.1.100:443 \\ 
TCP & (any) & 198.51.100.17:22 & 192.168.1.100:22 \\ 
\end{tabular}
\end{frame}

\begin{frame}<2>[label=servInNat]{running servers inside NAT (2)}
    \begin{itemize}
    \item often want to accept connections not configured by sysadmin
    \item example: direct video call between two users
        \begin{itemize}
        \item would be better to send directly
        \end{itemize}
    \vspace{.5cm}
    \item some classes of solution:
        \begin{itemize}
        \item \myemph<2>{ask router to add table entry}
        \item \myemph<3>{coordinate with other end to setup connection}
        \item \myemph<4>{go through realy}
        \end{itemize}
    \item extra issues:
        \begin{itemize}
        \item \myemph<5>{two hosts behind same NAT? nested NATs?}
    \end{itemize}
\end{frame}

\againframe<2>{servInNat}

\begin{frame}{router protocols}
    \begin{itemize}
    \item several (related) protocols for router-helped NAT traversal
        \begin{itemize}
        \item Port Control Protocol, NAT Port Mapping Protocol, UPnP Internet Gateway Protocol
        \end{itemze}
    \item discovered via UDP multicast and/or DHCP
    \item all provide:
        \begin{itemize}
        \item way of learning next external IP
        \item way to requesting externally accessible port
        \end{itemize}
    \item typical have lease times for external ports
        \begin{itemize}
        \item host is expected to renew periodically
        \end{itemize}
\end{frame}

\againframe<3>{servInNat}

\begin{frame}
\end{frame}

\againframe<4>{servInNat}

\begin{frame}{external relays}
    \begin{itemize}
    \end{itemize}
\end{frame}

\begin{frame}{
\end{frame}
