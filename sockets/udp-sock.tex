\begin{frame}[fragile]{unconnected UDP sockets (Python)}
\begin{Verbatim}[fontsize=\small]
import socket
# ipv4
sock = socket.socket(socket.AF_INET, socket.SOCK_DGRAM)
sock = socket.socket(socket.AF_INET, socket.SOCK_DGRAM,
                     socket.IPPROTO_UDP)
# ipv6
sock = socket.socket(socket.AF_INET6, socket.SOCK_DGRAM)
sock = socket.socket(socket.AF_INET6, socket.SOCK_DGRAM,
                     socket.IPPROTO_UDP)

sock.bind((local_host, local_port))
# local_host == "0.0.0.0" (v4) or "::" (v6) for "all possible"

data1, (host1, port1) = sock.recvfrom(MAX_SIZE)
data2, (host2, port2) = sock.recvfrom(MAX_SIZE)

sock.sendto(data, (host3, port3))
\end{Verbatim}
\end{frame}

\begin{frame}[fragile]{unconnected UDP sockets (Python, generic)}
\begin{Verbatim}[fontsize=\small]
import socket
possible_local_addrs = socket.getaddrinfo(
        local_host, local_port,
        family=socket.AF_UNSPEC,
        type=socket.SOCK_DGRAM,
        flags=socket.AI_PASSIVE)

# now need to handle that local_host might have multiple addresses
for af, socktype, proto, canonname, sockaddr in possible_local_addrs:
    sock = socket.socket(af, socktype, proto)
    sock.bind(sockaddr)
    
    # then use sock for recvfrom/sendto somehow
\end{Verbatim}
\end{frame}

\begin{frame}[fragile]{unconnected UDP sockets (Python, generic)}
\begin{Verbatim}[fontsize=\small]
import socket
possible_remote_addrs = socket.getaddrinfo(
        remote_host, remote_port,
        family=socket.AF_UNSPEC,
        type=socket.SOCK_DGRAM,
        flags=0)

for af, socktype, proto, canonname, sockaddr in possible_lremote_addrs:
    sock = socket.socket(af, socktype, proto)
    try:
        # hope local_host is acceptable for 'af'?
        # maybe choose local_host based on 'af' value?
        sock.bind((local_host, local_addr))
        sock.sendto(data, ...)
        ...
    except ...:
        ...
        # hope another address works?
        continue
\end{Verbatim}
\end{frame}

\begin{frame}{aside: getaddrinfo}
    \begin{itemize}
    \item lookup host+port by name
    \item can handle both IPv4 and IPv6
    \item AI\_PASSIVE = for bind()
    \item returns list of addresses
        \begin{itemize}
        \item sometimes: just choose one address (send message to server)
        \item sometimes: want to make socket for each address (run server)
        \end{itemize}
    \end{itemize}
\end{frame}

\begin{frame}[fragile]{unconnected UDP sockets (C, IPv4)}
\begin{Verbatim}[fontsize=\small]
int fd = socket(AF_INET, SOCK_DGRAM, 0 /* or IPPROTO_UDP */);
/* for 10.1.2.3 port 12345 
   there are utility functions to help with this, so you 
   shouldn't actually construct local_addr manually like this */
struct sockaddr_in local_addr;
memset(&local_addr, 0, sizeof(local_addr);
local_addr.sin_family = AF_INET;
local_addr.sin_port = htons(12345);
local_addr.sin_addr.s_addr = htonl((10<<24)|(1<<16)|(2<<8)|3),
bind(fd, &local_addr, sizeof(local_addr));
struct sockaddr_in remote1, remote2;
recvfrom(fd, buffer1, buffer1_size, 0, &remote1, sizeof(remote1));
recvfrom(fd, buffer2, buffer2_size, 0, &remote2, sizeof(remote2));
struct sockaddr_in remote3 = ...;
sendto(fd, buffer3, buffer3_size, 0, &remote3, sizeof(remote3));
\end{Verbatim}
\end{frame}

\begin{frame}[fragile]{connected UDP sockets (Python)}
\begin{Verbatim}[fontsize=\small]
import socket
# ipv4
sock = socket.socket(socket.AF_INET, socket.SOCK_DGRAM)
# ipv6
sock = socket.socket(socket.AF_INET6, socket.SOCK_DGRAM)

sock.bind((local_host, local_port))
sock.connect((remote_host, remote_port))
sock.send(b"one datagram")
sock.send(b"another datagram")
remote_data1 = sock.recv(MAX_SIZE)
remote_data2 = sock.recv(MAX_SIZE)
\end{Verbatim}
\end{frame}

\begin{frame}{choosing v4/v6?}
    \begin{itemize}
    \item nicer to write applications that handle both
    \item \ldots without explicitly doing this differently
    \vspace{.5cm}
    \item later socket API for that is getaddrinfo()
        \begin{itemize}
        \item also exists in C, but lots of cruft\ldots
        \end{itemize}
    \end{itemize}
\end{frame}

