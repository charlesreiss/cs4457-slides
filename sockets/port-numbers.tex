\begin{frame}{port numbers}
    \begin{itemize}
    \item we run multiple programs on a machine
        \begin{itemize}
        \item IP addresses identifying machine --- not enough
        \end{itemize}
    \item<2-> so, add 16-bit \textit{port numbers}
        \begin{itemize}
        \item<2-> think: multiple PO boxes at address
        \end{itemize}
    \vspace{.5cm}
    \item<3-> 0--49151: typically assigned for particular services
        \begin{itemize}
        \item 80 = http, 443 = https, 22 = ssh, \ldots
        \item usually <1024 reserved for sysadmin-only-use
        \end{itemize}
    \item<3-> 49152--65535: allocated on demand
        \begin{itemize}
        \item default ``return address'' for client connecting to server
        \end{itemize}
    \end{itemize}
\end{frame}


\begin{frame}{port number independence}
    \begin{itemize}
    \item typically ``5-tuple'' identifies socket: \\
        (protocol=TCP or UDP, source IP, dest IP, source port, dest port)
    \item means can have:
        \begin{itemize}
        \item two sockets with same source IP+source port, but different destinations
        \item two sockets with same dest IP+des port, but different sources
        \item two sockets with different protocols, but everything else same
        \end{itemize}
    \item special cases:
        \begin{itemize}
        \item dest IP/port can be `wildcard' (all zeroes usually)
        \item used for server that can be connected to
        \item used for unconnected UDP sockets
        \end{itemize}
    \end{itemize}
\end{frame}
