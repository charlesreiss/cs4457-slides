\begin{frame}[fragile,label=connSetupClientAddrInfo]{connection setup: client, using addrinfo}
\begin{lstlisting}[
    language=C++,style=smaller,
    moredelim={**[is][\btHL<2|handout:2>]{@2}{2@}},
    moredelim={**[is][\btHL<3|handout:3>]{@3}{3@}},
    moredelim={**[is][\btHL<4|handout:4>]{@4}{4@}},
    moredelim={**[is][\btHL<5|handout:5>]{@5}{5@}},
    moredelim={**[is][\btHL<6|handout:6>]{@6}{6@}},
]
int sock_fd; 
@2struct addrinfo2@ *server = /* code on next slide */;

sock_fd = socket(
    @3server->ai_family3@,  
     // ai_family = AF_INET (IPv4) or AF_INET6 (IPv6) or ...
    @3server->ai_socktype3@,
     // ai_socktype = SOCK_STREAM (bytes) or ...
    @3server->ai_prototcol3@
     // ai_protocol = IPPROTO_TCP or ...
);
if (sock_fd < 0) { /* handle error */ }
if (connect(sock_fd, server->@4ai_addr4@, server->@4ai_addrlen4@) < 0) {
    /* handle error */
}
@5freeaddrinfo(server);5@
DoClientStuff(sock_fd); /* read and write from sock_fd */
close(sock_fd);
\end{lstlisting}
\begin{tikzpicture}[overlay,remember picture]
\tikzset{
    box/.style={draw=red, ultra thick,fill=white,at={([yshift=-2cm]current page.north)},anchor=north,align=left},
    box low/.style={draw=red, ultra thick,fill=white,at={([yshift=-5cm]current page.north)},anchor=north,align=left},
}
\begin{visibleenv}<2>
\node[box low] {
    addrinfo contains all information needed to setup socket \\
    set by \texttt{getaddrinfo} function (next slide) \\
    handles IPv4 and IPv6 \\
    handles DNS names, service names
};
\end{visibleenv}
\begin{visibleenv}<4>
\node[box] {
    ai\_addr points to struct representing address \\
    type of struct depends whether IPv6 or IPv4
};
\end{visibleenv}
\begin{visibleenv}<5>
\node[box] {
    since addrinfo contains pointers to dynamically allocated memory, \\
    call this function to free everything
};
\end{visibleenv}
\end{tikzpicture}
\end{frame}

\begin{frame}[fragile,label=serverLookupClient]{connection setup: lookup address}
\begin{lstlisting}[
    language=C++,style=smaller,
    moredelim={**[is][\btHL<2|handout:2>]{@2}{2@}},
    moredelim={**[is][\btHL<3|handout:3>]{@3}{3@}},
    moredelim={**[is][\btHL<4|handout:4>]{@4}{4@}},
    moredelim={**[is][\btHL<5|handout:5>]{@5}{5@}},
    moredelim={**[is][\btHL<6|handout:6>]{@6}{6@}},
]
/* example hostname, portname = "www.cs.virginia.edu", "443" */
const char *hostname; const char *portname;
...
struct addrinfo *server;
struct addrinfo hints;
int rv;
memset(&hints, 0, sizeof(hints));
@3hints.ai_family = AF_UNSPEC3@; /* for IPv4 OR IPv6 */
// hints.ai_family = AF_INET4; /* for IPv4 only */

hints.ai_socktype = SOCK_STREAM; /* byte-oriented --- TCP */
rv = getaddrinfo(hostname, portname, &hints, @2&server2@);
if (rv != 0) { /* handle error */ }

/* eventually freeaddrinfo(result) */
\end{lstlisting}
\begin{tikzpicture}[overlay,remember picture]
\tikzset{
    box/.style={draw=red, ultra thick,fill=white,at={([yshift=-2cm]current page.north)},anchor=north,align=left},
    box low/.style={draw=red, ultra thick,fill=white,at={([yshift=-5cm]current page.north)},anchor=north,align=left},
}
\begin{visibleenv}<2>
\node[box low] {
    NB: pass pointer \textit{to pointer} to addrinfo to fill in
};
\end{visibleenv}
\begin{visibleenv}<3>
\node[box] {
    AF\_UNSPEC: choose between IPv4 and IPv6 for me \\
    AF\_INET, AF\_INET6: choose IPv4 or IPV6 respectively
};
\end{visibleenv}
\end{tikzpicture}
\end{frame}
