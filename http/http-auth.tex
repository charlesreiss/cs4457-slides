\begin{frame}{HTTP authentication (RFC 7235)}
    \begin{itemize}
    \item relatively uncommon today compared to cookies (next topic)
    \vspace{.5cm}
    \item server$\rightarrow$ client: 401 Unauthorized with \textit{WWW-Authenticate} header
        \begin{itemize}
        \item typically: web browser shows login prompt
        \end{itemize}
    \item client$\rightarrow$ server: Authorization header
        \begin{itemize}
            \item example: \texttt{Authorization: Basic dGVzdDoxMjPCow==}
            \item \texttt{test:123} in Base64 (username test, password 123)
            \item example: \texttt{Authorization: Bearer mF\_9.B5f-4.1JqM}
            \item bearer `token' obtained \myemph<2>{out of band} (common in application `APIs')
        \end{itemize}
    \end{itemize}
\begin{tikzpicture}[overlay,remember picture]
\begin{visibleenv}<2>
    \node[anchor=north,draw=red,ultra thick,fill=white] (token) at ([yshift=-1cm]current page.north) {
    \includegraphics[height=0.6\textheight,alt={Canvas's dialog labeled `New Access Token'. Says `Access tokens are what allow third-party applications to access Canvas resources on your benhalf. These tokens are normally created autommatically for applications as needed, ut if you're developing a new or limited project you can just generate the token from here.' and has a prompt for information and a `Generate Token' button.}]{../http/http-bearer-make-canvas}
};
\end{visibleenv}
\end{tikzpicture}
\end{frame}
