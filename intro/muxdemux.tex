

\begin{frame}\frametitle{(de)multiplexing}
\myalttext{
\begin{tikzpicture}
\tikzset{%
    connect one/.style={draw,very thick,-Latex},
    connect one lg/.style={draw,line width=1mm,-Latex},
    connect one sm/.style={draw,thick,-Latex},
    computer/.style={inner sep=0mm,outer sep=0mm,execute at begin node={\computer}},
    switch/.style={inner sep=0mm,outer sep=0mm,execute at begin node={\switch}},
    big switch/.style={inner sep=0mm,outer sep=0mm,execute at begin node={\bigswitch}},
    packet/.style={minimum width=.4cm,minimum height=0.2cm,inner sep=0mm,outer sep=0mm,draw},
    packet lg/.style={minimum width=.6cm,minimum height=0.2cm,inner sep=0mm,outer sep=0mm,draw},
    c1c2/.style={fill=violet!40,draw=black,thin},
    c3c4/.style={pattern=checkerboard,pattern color=green!70,draw=black,thin},
    buffer 1/.style={slidealt=<5>{draw=red,thick},slidealt=<8>{draw=red,thick}},
    buffer 2/.style={slidealt=<5>{draw=red,thick}},
}
\node[computer] (c1) at (-5, 1) {};
\node at (c1) {1};
\node[big switch,slidealt=<3-4>{fill=red!10},slidealt=<6>{fill=red!10}] (s1) at (-2,-.5) {};
\node[big switch,slidealt=<3-4>{fill=red!10}] (s2) at (2,.5) {};
\node[computer] (c2) at (6, 2) {};
\node at (c2) {2};
\node[computer] (c3) at (-5, -1) {};
\node at (c3) {3};
\node[computer] (c4) at (6, 0) {};
\node at (c4) {4};
\draw[connect one] (c3) -- (s1)
    node[above=0.05cm,sloped,packet lg,c3c4,pos=0.2] {}
    node[above=0.05cm,sloped,packet,c3c4,pos=0.7] {};
\draw[connect one] (c1) -- (s1)
    node[above=0.05cm,sloped,packet,c1c2,pos=0.1] {}
    node[above=0.05cm,sloped,packet lg,c1c2,pos=0.5] {}
    node[above=0.05cm,sloped,packet,c1c2,pos=0.8] {};
\draw[connect one,slidealt={<2>{draw=red,ultra thick}}] (s1) -- (s2)
    node[above=0.05cm,sloped,packet,c1c2,pos=0.3] {}
    node[above=0.05cm,sloped,packet,c3c4,pos=0.6] {}
    node[above=0.05cm,sloped,packet,c1c2,pos=0.8] {};
\draw[connect one sm] (s2.north east) -- (c2)
    node[above=0.05cm,sloped,packet,c1c2,pos=0.2] {}
    node[above=0.05cm,sloped,packet lg,c1c2,pos=0.6] {};
\draw[connect one sm] (s2.south east) -- (c4)
    node[above=0.05cm,sloped,packet,c3c4,pos=0.2] {}
    node[above=0.05cm,sloped,packet lg,c3c4,pos=0.6] {};
\node[packet,c3c4,buffer 1,anchor=north east] at ([yshift=-.1cm,xshift=-.1cm]s1.north east) {};
\node[packet,c1c2,buffer 1,anchor=north east] at ([yshift=-.3cm,xshift=-.1cm]s1.north east) {};
\node[packet lg,c1c2,buffer 1,anchor=north east] at ([yshift=-.5cm,xshift=-.1cm]s1.north east) {};
\begin{visibleenv}<8>
\node[packet,c3c4,buffer 1,anchor=north east,opacity=0.9,draw=red] at ([yshift=-.7cm,xshift=-.1cm]s1.north east) {};
\node[packet lg,c1c2,buffer 1,anchor=north east,opacity=0.8,draw=red] at ([yshift=-.9cm,xshift=-.1cm]s1.north east) {};
\node[packet,c1c2,buffer 1,anchor=north east,opacity=0.7,draw=red] at ([yshift=-1.1cm,xshift=-.1cm]s1.north east) {};
\node[packet,c1c2,buffer 1,anchor=north east,opacity=0.5,draw=red] at ([yshift=-1.3cm,xshift=-.1cm]s1.north east) {};
\node[packet,c1c2,buffer 1,anchor=north east,opacity=0.3,draw=red] at ([yshift=-1.3cm,xshift=-.1cm]s1.north east) {};
\node[packet,c1c2,buffer 1,anchor=north east,opacity=0.1,draw=red] at ([yshift=-1.5cm,xshift=-.1cm]s1.north east) {};
\end{visibleenv}
\node[packet,c1c2,buffer 2,anchor=north east] at ([yshift=-.1cm,xshift=-.1cm]s2.north east) {};
\node[packet,c1c2,buffer 2,anchor=north east] at ([yshift=-.3cm,xshift=-.1cm]s2.north east) {};
\node[packet,c1c2,buffer 2,anchor=north east] at ([yshift=-.5cm,xshift=-.1cm]s2.north east) {};
\node[packet lg,c3c4,buffer 2,anchor=south east] at ([yshift=.1cm,xshift=-.1cm]s2.south east) {};
\coordinate (box loc) at (-4, -1.5);
\tikzset{%
    explain box/.style={%
        overlay,draw=red, align=left, very thick, anchor=north west,at=(box loc)
    },
}
\begin{visibleenv}<2>
\node[explain box] {%
    two or more flows can \\
    share one or more links
};
\end{visibleenv}
\begin{visibleenv}<3>
\node[explain box] {%
    left switch \textit{\myemph{multiplexes}} the two flows onto one link \\
    right switch \textit{\myemph{demultiplexes}} them to separate them
};
\end{visibleenv}
\begin{visibleenv}<4>
\node[explain box] {%
    this picture: multiplexed by dividing up \textit{time} on link
};
\end{visibleenv}
\begin{visibleenv}<5>
\node[explain box] {%
    switches usually have \textit{\myemph{buffers}} (also called \textit{\myemph{queues}}) \\
    hold waiting packets \\
    ~ \\
    absorbs temporary ``bursts'' where packets come faster \\
    than outgoing link can handle
        % FIXME: diagram of packets coming in over time
};
\end{visibleenv}
\end{tikzpicture}
}{
Diagram showing two flows (on between machines 1 and 2; the other between machines 3 and 4), each divided up into packets.
The flows both pass from the source machine, to a first switch, then to a second switch, then to the destination machine.
Packets from both flows are interleaved on the link connnecting the two switches.
Interleaving the packets is called *multiplexing* and deinterleaving the packets is called *demultiplexing*. This diagram
shows packets multiplexed in time, but other mechanisms are possible.
The switches have *buffers* (or *queues*) that hold packets.
}
\end{frame}

\begin{frame}<6->\frametitle{bursts and dropping}
\myalttext{
\begin{tikzpicture}
\tikzset{%
    connect one/.style={draw,very thick,-Latex},
    connect one lg/.style={draw,line width=1mm,-Latex},
    connect one sm/.style={draw,thick,-Latex},
    computer/.style={inner sep=0mm,outer sep=0mm,execute at begin node={\computer}},
    switch/.style={inner sep=0mm,outer sep=0mm,execute at begin node={\switch}},
    big switch/.style={inner sep=0mm,outer sep=0mm,execute at begin node={\bigswitch}},
    packet/.style={minimum width=.4cm,minimum height=0.2cm,inner sep=0mm,outer sep=0mm,draw},
    packet lg/.style={minimum width=.6cm,minimum height=0.2cm,inner sep=0mm,outer sep=0mm,draw},
    c1c2/.style={fill=violet!40,draw=black,thin},
    c3c4/.style={pattern=checkerboard,pattern color=green!70,draw=black,thin},
    buffer 1/.style={slidealt=<5>{draw=red,thick},slidealt=<8>{draw=red,thick}},
    buffer 2/.style={slidealt=<5>{draw=red,thick}},
}
\node[computer] (c1) at (-5, 1) {};
\node at (c1) {1};
\node[big switch,slidealt=<3-4>{fill=red!10},slidealt=<6>{fill=red!10}] (s1) at (-2,-.5) {};
\node[big switch,slidealt=<3-4>{fill=red!10}] (s2) at (2,.5) {};
\node[computer] (c2) at (6, 2) {};
\node at (c2) {2};
\node[computer] (c3) at (-5, -1) {};
\node at (c3) {3};
\node[computer] (c4) at (6, 0) {};
\node at (c4) {4};
\draw[connect one] (c3) -- (s1)
    node[above=0.05cm,sloped,packet lg,c3c4,pos=0.2] {}
    node[above=0.05cm,sloped,packet,c3c4,pos=0.7] {};
\draw[connect one] (c1) -- (s1)
    node[above=0.05cm,sloped,packet,c1c2,pos=0.1] {}
    node[above=0.05cm,sloped,packet lg,c1c2,pos=0.5] {}
    node[above=0.05cm,sloped,packet,c1c2,pos=0.8] {};
\draw[connect one,slidealt={<2>{draw=red,ultra thick}}] (s1) -- (s2)
    node[above=0.05cm,sloped,packet,c1c2,pos=0.3] {}
    node[above=0.05cm,sloped,packet,c3c4,pos=0.6] {}
    node[above=0.05cm,sloped,packet,c1c2,pos=0.8] {};
\draw[connect one sm] (s2.north east) -- (c2)
    node[above=0.05cm,sloped,packet,c1c2,pos=0.2] {}
    node[above=0.05cm,sloped,packet lg,c1c2,pos=0.6] {};
\draw[connect one sm] (s2.south east) -- (c4)
    node[above=0.05cm,sloped,packet,c3c4,pos=0.2] {}
    node[above=0.05cm,sloped,packet lg,c3c4,pos=0.6] {};
\node[packet,c3c4,buffer 1,anchor=north east] at ([yshift=-.1cm,xshift=-.1cm]s1.north east) {};
\node[packet,c1c2,buffer 1,anchor=north east] at ([yshift=-.3cm,xshift=-.1cm]s1.north east) {};
\node[packet lg,c1c2,buffer 1,anchor=north east] at ([yshift=-.5cm,xshift=-.1cm]s1.north east) {};
\begin{visibleenv}<8>
\node[packet,c3c4,buffer 1,anchor=north east,opacity=0.9,draw=red] at ([yshift=-.7cm,xshift=-.1cm]s1.north east) {};
\node[packet lg,c1c2,buffer 1,anchor=north east,opacity=0.8,draw=red] at ([yshift=-.9cm,xshift=-.1cm]s1.north east) {};
\node[packet,c1c2,buffer 1,anchor=north east,opacity=0.7,draw=red] at ([yshift=-1.1cm,xshift=-.1cm]s1.north east) {};
\node[packet,c1c2,buffer 1,anchor=north east,opacity=0.5,draw=red] at ([yshift=-1.3cm,xshift=-.1cm]s1.north east) {};
\node[packet,c1c2,buffer 1,anchor=north east,opacity=0.3,draw=red] at ([yshift=-1.3cm,xshift=-.1cm]s1.north east) {};
\node[packet,c1c2,buffer 1,anchor=north east,opacity=0.1,draw=red] at ([yshift=-1.5cm,xshift=-.1cm]s1.north east) {};
\end{visibleenv}
\node[packet,c1c2,buffer 2,anchor=north east] at ([yshift=-.1cm,xshift=-.1cm]s2.north east) {};
\node[packet,c1c2,buffer 2,anchor=north east] at ([yshift=-.3cm,xshift=-.1cm]s2.north east) {};
\node[packet,c1c2,buffer 2,anchor=north east] at ([yshift=-.5cm,xshift=-.1cm]s2.north east) {};
\node[packet lg,c3c4,buffer 2,anchor=south east] at ([yshift=.1cm,xshift=-.1cm]s2.south east) {};
\coordinate (box loc) at (-4, -1.5);
\tikzset{%
    explain box/.style={%
        overlay,draw=red, align=left, very thick, anchor=north west,at=(box loc)
    },
}
\begin{visibleenv}<6-7>
\node[explain box] {%
    incomplete list of causes of `bursts': \\
    ~ \\
    \myemph<6>{multiple unsynchronized flows} \\
    \myemph<7>{fast links produce packets faster for slow can send}
};
\end{visibleenv}
\begin{visibleenv}<8>
\node[explain box] {%
    if buffer full, switch must \textit{\myemph{drop}} packets \\
    will happen eventually if overall rate faster than outgoing link \\
    ~\\
    scenario is called \textit{\myemph{congestion}}
};
\end{visibleenv}
\end{tikzpicture}
}{
Same diagram as previous slide, but with more packets in switch queues, eventually overflowing.
Some possible causes of bursts are: multiple unsynchronized flows, or fast links producing packets faster than a slower can send. 
When the buffer is not large enough to handle the burst, the switch will *drop* packets. This scenario is called *congestion*.
}
\end{frame}

