
\begin{frame}\frametitle{buffer usage: fast to slow, store + forward}
\myalttext{
\begin{tikzpicture}
\tikzset{
    axis/.style={draw,thick,-Latex},
    scale mark/.style={draw,thin},
    scale label/.style={font=\small},
    packet/.style={ultra thick,fill=violet!20},
    y=.6cm
}
\begin{scope} % input
    \begin{scope}[shift={(0.1, 0)}]
        \clip (0, 0) rectangle (12.5, 3.2);
        \draw[packet] (0, 0) coordinate (A recv start) rectangle (2, 3) node[midway] {packet A};
        \draw[packet] (2, 0) rectangle (4, 3) node[midway] {packet B};
        \draw[packet] (4, 0) rectangle (6, 3) node[midway] {packet C};
        \draw[packet] (10, 0) rectangle (12, 3) node[midway] {packet D};
    \end{scope}
    \draw[axis] (0, 0) -- ++ (12.5, 0);
    \draw[axis] (0, 0) -- ++ (0, 3.3)
        node[midway,left=.5cm,align=right] { input };
    \draw[scale mark] (0, 3) -- ++ (.25, 0) node[pos=0,left,scale label] {capacity};
\end{scope} 
\begin{scope}[shift={(0, -5)}]% buffer usage
    \begin{scope}
        \clip (0,0) rectangle (12.5, 4);
        \draw[violet, ultra thick] (0, 0) -- (0.1, 0) -- (0.1, 1) -- (2.1, 1) -- (2.1, 2) -- (4.1, 2) -- (4.1, 3)
            -- (5.1, 3) -- (5.1, 2) -- (8.1, 2) -- (8.1, 1) -- (10.1, 1) -- (10.1, 2) -- 
            (11.1, 2) -- (11.1, 1) -- (14, 1);
    \end{scope}
    \draw[axis] (0, 0) -- ++ (12.5, 0);
    \draw[axis] (0, 0) -- ++ (0, 4.3)
    node[midway,left=.5cm,align=right] (usage label) { buffer \\ reserved };
    \node[font=\small,anchor=north,align=center] at (usage label.south) {
        packets
    };
    \draw[scale mark] (0, 1) -- ++ (.25, 0) node[pos=0,left,scale label] {1};
    \draw[scale mark] (0, 2) -- ++ (.25, 0) node[pos=0,left,scale label] {2};
    \draw[scale mark] (0, 3) -- ++ (.25, 0) node[pos=0,left,scale label] {3};
    \draw[scale mark] (0, 4) -- ++ (.25, 0) node[pos=0,left,scale label] {4};
%
\end{scope}
\begin{scope}[shift={(0, -8)}] % output
    \begin{scope}
        \clip (0, 0) rectangle (12.5, 2.2);
        \begin{scope}[shift={(2.1, 0)}]
            \draw[packet] (0, 0) coordinate (A send start) rectangle (3, 2) node[midway] {packet A};
            \draw[packet] (3, 0) rectangle (6, 2) node[midway] {packet B};
            \draw[packet] (6, 0) rectangle (9, 2) node[midway] {packet C};
            \draw[packet] (10, 0) rectangle (13, 2) node[midway] {packet D};
        \end{scope}
    \end{scope}
    \draw[axis] (0, 0) -- ++ (12.5, 0);
    \draw[axis] (0, 0) -- ++ (0, 2.3)
        node[midway,left=.5cm,align=right] { output };
    \draw[scale mark] (0, 2) -- ++ (.25, 0) node[pos=0,left,scale label] {capacity};
\end{scope}
\begin{visibleenv}<2>
    \draw[draw=red,dotted,ultra thick,Latex-Latex] (A send start) -- ++(-2cm, 0);
    \draw[draw=red,dotted,ultra thick] (A send start) -- ++(0cm, 6cm);
    \draw[draw=red,dotted,ultra thick] ([xshift=-2cm]A send start) -- ++(0cm, 6cm);
    \node[draw=red,ultra thick,align=left,fill=white,anchor=west] at  (2.2, -3) {
        \textit{store and forward} \\
        switch stores whole packet in buffer\\
        then sends it out \\
        ~ \\
        our default in this class
    };
\end{visibleenv}
\end{tikzpicture}
}{
*store and forward* switches are illustrated as taking in packets and buffering them
as they received, and only sending packets after they are fully contained in the buffer.
These are the most common type of switches, by far.
}
\end{frame}

\begin{frame}\frametitle{buffer usage: fast to slow, cut-through}
\myalttext{
\begin{tikzpicture}
\tikzset{
    axis/.style={draw,thick,-Latex},
    scale mark/.style={draw,thin},
    scale label/.style={font=\small},
    packet/.style={ultra thick,fill=violet!20},
    y=.6cm
}
\begin{scope} % input
    \begin{scope}[shift={(0.1, 0)}]
        \clip (0, 0) rectangle (12.5, 3.2);
        \draw[packet] (0, 0) coordinate (A recv start) rectangle (2, 3) node[midway] {packet A};
        \draw[packet] (2, 0) rectangle (4, 3) node[midway] {packet B};
        \draw[packet] (4, 0) rectangle (6, 3) node[midway] {packet C};
        \draw[packet] (10, 0) rectangle (12, 3) node[midway] {packet D};
    \end{scope}
    \draw[axis] (0, 0) -- ++ (12.5, 0);
    \draw[axis] (0, 0) -- ++ (0, 3.3)
        node[midway,left=.5cm,align=right] { input };
    \draw[scale mark] (0, 3) -- ++ (.25, 0) node[pos=0,left,scale label] {capacity};
\end{scope} 
\begin{scope}[shift={(0, -5)}]% buffer usage
    \begin{scope}
        \clip (0,0) rectangle (12.5, 4);
        \draw[violet, ultra thick] (0, 0) -- (0.1, 0) -- (0.1, 1) -- (2.1, 1) -- (2.1, 2) -- 
            (3.6, 2) -- (3.6, 1) -- (4.1, 1) -- (4.1, 2) -- (6.6, 2) -- (6.6, 1) --
            (9.6, 1) -- (9.6, 0) -- (10.1, 0) -- (10.1, 1) -- (13.6, 1) -- (13.6, 0);
    \end{scope}
    \draw[axis] (0, 0) -- ++ (12.5, 0);
    \draw[axis] (0, 0) -- ++ (0, 4.3)
    node[midway,left=.5cm,align=right] (usage label) { buffer \\ reserved };
    \node[font=\small,anchor=north,align=center] at (usage label.south) {
        packets
    };
    \draw[scale mark] (0, 1) -- ++ (.25, 0) node[pos=0,left,scale label] {1};
    \draw[scale mark] (0, 2) -- ++ (.25, 0) node[pos=0,left,scale label] {2};
    \draw[scale mark] (0, 3) -- ++ (.25, 0) node[pos=0,left,scale label] {3};
    \draw[scale mark] (0, 4) -- ++ (.25, 0) node[pos=0,left,scale label] {4};
%
\end{scope}
\begin{scope}[shift={(0, -8)}] % output
    \begin{scope}
        \clip (0, 0) rectangle (12.5, 2.2);
        \begin{scope}[shift={(0.6, 0)}]
            \draw[packet] (0, 0) coordinate (A send start) rectangle (3, 2) node[midway] {packet A};
            \draw[packet] (3, 0) rectangle (6, 2) node[midway] {packet B};
            \draw[packet] (6, 0) rectangle (9, 2) node[midway] {packet C};
            \draw[packet] (10, 0) rectangle (13, 2) node[midway] {packet D};
        \end{scope}
    \end{scope}
    \draw[axis] (0, 0) -- ++ (12.5, 0);
    \draw[axis] (0, 0) -- ++ (0, 2.3)
        node[midway,left=.5cm,align=right] { output };
    \draw[scale mark] (0, 2) -- ++ (.25, 0) node[pos=0,left,scale label] {capacity};
\end{scope}
\begin{visibleenv}<1>
    \draw[draw=red,dotted,ultra thick] (A send start) -- ++(-.5cm, 0);
    \draw[draw=red,dotted,ultra thick] (A send start) -- ++(0cm, 6cm);
    \draw[draw=red,dotted,ultra thick] ([xshift=-.5cm]A send start) -- ++(0cm, 6cm);
    \node[draw=red,ultra thick,align=left,fill=white,anchor=west] at  (2.2, -3) {
        \textit{cut-through} forwarding \\
        switch sends packet out as it's being received \\
        ~ \\
        uncommon and much more complex to implement
    };
\end{visibleenv}
\end{tikzpicture}
}{
*cut-through* switches are illustrated as sending packets as they are being received
as long as no other packets are waiting. Though more efficient,
these are uncommon as they are more difficult to implement than
store-and-forward switches (because of, for example, error handling)
}
\end{frame}
% FIXME: multiplexing at end hosts
