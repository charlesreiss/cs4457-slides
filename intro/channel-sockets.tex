\begin{frame}{channel abstraction and sockets}
    \begin{itemize}
    \item BSD \textit{sockets} are most used abstract for using channels
    \item \texttt{channel} is called \texttt{connection}
    \item server (passive end)
        \begin{itemize}
        \item create socket (\texttt{socket()})
        \item select address (\texttt{bind()})
        \item wait for+get connection (\texttt{listen()}+\texttt{accept()})
        \item read+write on connection(\texttt{read()}+\texttt{recv*()}+\texttt{write()}+\texttt{send*()})
        \end{itemize}
    \item client (active end)
        \begin{itemize}
        \item create socket (\texttt{socket()})
        \item connect to address (\texttt{connect()}
        \item read+write on connection(\texttt{read()}+\texttt{recv*()}+\texttt{write()}+\texttt{send*()})
        \end{itemize}
    \end{itemize}
\end{frame}

\begin{frame}{sockets and other options}
    \begin{itemize}
    \item sockets can also provide \textit{datagram} abstraction
        \begin{itemize}
        \item difference: mode where read/write keeps messages together
        \end{itemize}
    \end{itemize}
\end{frame}

\begin{frame}{socket details later}
    \begin{itemize}
    \item we're doing mostly bottom-up approach
    \item will actually talk in detail about socket interface later in semester
    \end{itemize}
\end{frame}
